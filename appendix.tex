
%\newpage
\appendix
The Appendix is organized as follows. First we prove some technical lemmas in Appendix \ref{App:Lemma1} and Appendix \ref{App:Lemma2}. Next we prove the main theorem in Appendix \ref{sec:proofTheorem}. In Appendix \ref{App:A} we prove Proposition \ref{proofTheorem:Prop:1}. In Appendix \ref{App:Proof:Corollary:1} we prove Corollary \ref{Result:Corollary:1}.

% and in Appendix \ref{App:Proof:Corollary:2} we prove Corollary \ref{Result:Corollary:2}. Appendix \ref{App:E} deals with the idea of why we do clustering. The simple regret bound of EClusUCB and its associated Corollary is proved in \ref{App:SR_EClusUCB}. Algorithm \ref{alg:aclusucb}, Adaptive Clustered UCB is shown in Appendix \ref{App:AClusUCB}. More experiments are shown in Appendix \ref{App:MoreExp}.

%\section{Regret Bound Table}
%\label{App:Table}
%\begin{table}[!h]
%\caption{Gap-dependent regret bounds for different bandit algorithms}
%\label{tab:regret-bds}
%\begin{center}
%\begin{tabular}{|c|c|}
%\toprule
%Algorithm  & Upper bound \\
%\midrule
%UCB1         &$O\left(\frac{K\log T}{\Delta}\right)$ \\\midrule
%UCB-Improved &$O\left(\frac{K\log (T\Delta^{2})}{\Delta}\right)$ \\\midrule
%MOSS	     &$O\left(\frac{K^{2}\log\left(T\Delta^{2}/K\right)}{\Delta}\right)$\\\midrule
%EClusUCB      &$O\left(\frac{K\log\left(\frac{T\Delta^{2}}{\sqrt{\log (K)}}\right)}{\Delta}\right)$\\\bottomrule
%\end{tabular}
%\end{center}
%\vspace*{-2em}
%\end{table}

\section{Proof of Lemma 1}
\label{App:Lemma1}

\begin{lemma}
\label{proofTheorem:Lemma:1}
If $T\geq K^{2.4}$, $\psi=\dfrac{T}{ K^2}$, $\rho_a =\dfrac{1}{2}$ and $m\leq \dfrac{1}{2} \log_2\left(\dfrac{T}{e}\right) $, then,
\begin{align*}
\dfrac{\rho_a m \log(2)}{\log(\psi T) - 2m\log( 2)} \leq \frac{3}{2}
\end{align*}
\end{lemma}

\begin{proof}
The proof is based on contradiction. Suppose
\begin{eqnarray*}
\dfrac{\rho m \log(2)}{\log(\psi T) - 2m\log( 2)} > \frac{3}{2}.
\end{eqnarray*}
Then, with $\psi=\dfrac{T}{ K^2}$ and $\rho_a =\dfrac{1}{2}$, we obtain
\begin{eqnarray*}
6\log(K) 
&>& 6\log(T) - 7m\log(2) \\
&\overset{(a)}{\ge}& 6\log(T) - \frac{7}{2} \log_2\left(\frac{T}{e}\right) \log(2) \\
&=& 2.5\log(T) + 3.5 \log_2(e)\log(2)  \\
&\overset{(b)}{=}& 2.5\log(T) +3.5
\end{eqnarray*}
where $(a)$ is obtained using $m\leq \dfrac{1}{2} \log_2\left(\dfrac{T}{e}\right)$, while $(b)$ follows from the identity $\log_2(e)\log(2) =1$. Finally, for $T\ge K^{2.4}$ we obtain, $6\log(K)>6\log(K)+3.5$, which is a contradiction. Hence, for $T\geq K^{2.4}$, $\psi=\dfrac{T}{ K^2}$, $\rho=\dfrac{1}{2}$ and $m \leq \dfrac{1}{2} \log_2\left(\dfrac{T}{e}\right) $ we have, 
\begin{align*}
\dfrac{\rho m \log(2)}{\log(\psi T) - 2m\log( 2)} \leq \frac{3}{2}
\end{align*}
\end{proof}

\section{Proof of Lemma 2}
\label{App:Lemma2}
\begin{lemma}
\label{proofTheorem:Lemma:2}
If $T\geq K^{2.4}$, $\psi=\dfrac{T}{ K^2}$, $\rho_a =\dfrac{1}{2}$, $m_i = min\lbrace m|\sqrt{2\epsilon_{m} } < \dfrac{\Delta_i}{4} \rbrace $ and $c_{m_i} =\sqrt{\frac{\rho_{a}\log (\psi T\epsilon_{m_{i}})}{2 n_{m_i}}}$, then, $c_{m_i} < \dfrac{\Delta_i}{4}$.
\end{lemma}

\begin{proof}
In the $m_i$-th round $c_{m_i} =\sqrt{\frac{\rho_{a}\log (\psi T\epsilon_{m_{i}})}{2 n_{m_i}}}$. Substituting the value of $n_{m_i}=\dfrac{\log{(\psi T\epsilon_{m_{i}}^{2})}}{2\epsilon_{m_{i}}}$ in $c_{m_i}$ we get,
\begin{align*}
	c_{m_i} &\leq \sqrt{\dfrac{\rho_a \epsilon_{m_{i}}\log (\psi T\epsilon_{m_{i}})}{\log(\psi T\epsilon_{m_{i}}^{2})}} \leq \sqrt{\dfrac{\rho_a \epsilon_{m_{i}}\log (\frac{\psi T\epsilon_{m_{i}}^{2}}{\epsilon_{m_{i}}})}{\log(\psi T\epsilon_{m_{i}}^{2})}} \\
	%%%%%%%%%%%%%%%%%%%%%%%%%%
	& = \sqrt{\dfrac{\rho_a\epsilon_{m_{i}}\log (\psi T\epsilon_{m_{i}}^{2}) - \rho_a\epsilon_{m_{i}}\log (\epsilon_{m_{i}})}{\log(\psi T\epsilon_{m_{i}}^{2})}} 
	\leq  \sqrt{\rho_a\epsilon_{m_{i}} - \dfrac{\rho_a\epsilon_{m_i}\log(\frac{1}{2^{m_i}})}{\log(\psi T \frac{1}{2^{2m_i}})}} \\
	%%%%%%%%%%%%%%%%%%%%%%%%%%
	&\leq \sqrt{\rho_a\epsilon_{m_{i}} + \dfrac{\rho_a\epsilon_{m_i}\log(2^{m_i})}{\log(\psi T) - \log( 2^{2m_i})}}  \leq \sqrt{\rho_a\epsilon_{m_{i}} + \dfrac{\rho_a\epsilon_{m_i}m_i \log(2)}{\log(\psi T) - 2m_i\log( 2)}} \\ 
	%%%%%%%%%%%%%%%%%%%%%%%%%%
	 & \overset{(a)}{\leq} \sqrt{\rho_a\epsilon_{m_{i}} + \frac{3}{2}\epsilon_{m_i}} 
	  < \sqrt{2\epsilon_{m_i}} 
	  < \dfrac{\Delta_{i}}{4} 
	\end{align*}
In the above simplification, $(a)$ is obtained using Lemma~\ref{proofTheorem:Lemma:1}. 
\end{proof}



\section{Proof of Theorem 1}
\label{sec:proofTheorem}
\begin{proof}
Let $A^{'}=\lbrace i \in A,\Delta_{i}> b\rbrace$,  $A^{''}=\lbrace i \in A, \Delta_{i} > 0\rbrace$, $A^{'}_{s_{k}}=\lbrace i \in A_{s_{k}},\Delta_{i}> b\rbrace$ and $A^{''}_{s_{k}}=\lbrace i \in A_{s_{k}}, \Delta_{i} > 0 \rbrace$. $C_{g}$ is the cluster set containing max payoff arm from each cluster in $g$-th round. The arm having the true highest payoff in a cluster $s_{k}$ is denote by $a_{\max_{s_{k}}}$. Let for each sub-optimal arm ${i}\in A$, $m_{i}=\min{\lbrace m|\sqrt{2\epsilon_{m}} < \frac{\Delta_{i}}{4} \rbrace}$ and let for each cluster $s_{k}\in S$, $g_{s_{k}}=\min{\lbrace g|\sqrt{2\epsilon_{g}} < \frac{\Delta_{a_{\max_{s_{k}}}}}{4} \rbrace}$. Let $\check{A}=\lbrace {i}\in A^{'} | {i}\in s_{k} , \forall s_{k}\in S \rbrace$. The analysis proceeds by considering the contribution to the regret in each of the following cases:

\textbf{Case a:} \textit{Some sub-optimal arm ${i}$ is not eliminated in round $\max(m_{i},g_{s_{k}})$ or before, with the optimal arm ${*}\in C_{\max(m_{i},g_{s_{k}})}$.}
We consider an arbitrary sub-optimal arm ${i}$ and analyze the contribution to the regret when $i$ is not eliminated in the following exhaustive sub-cases:\\
\textbf{Case a1:} \textit{In round $\max(m_{i},g_{s_{k}})$, ${i} \in s^{*}$.}

Similar to case (a) of \cite{auer2010ucb}, observe that when the following two conditions hold, arm $i$ gets eliminated:
\begin{align}
\hat{r}_{i}  \le r_{i} + c_{m_i} \text{ and } 
 \hat{r}^{*}\geq  r^{*} - c_{m_i}, \label{eq:armelim-casea1}
\end{align}
where  $c_{m_i} = \sqrt{\frac{\rho_{a}\log (\psi T\epsilon_{m_{i}})}{2 n_{m_i}}}$. The arm $i$ gets eliminated because 
  \begin{align*}
\hat{r}_{i} +c_{m_i} & \leq r_{i} + 2c_{m_i} < r_{i} + \Delta_{i} - 2c_{m_i}\\
 &\leq r^{*} -2c_{m_i} \leq \hat{r}^{*} - c_{m_i}  .
  \end{align*}
In the above, we have used the fact that $ c_{m_i} = \sqrt{\epsilon_{m_{i}+1}} < \frac{\Delta_{i}}{4}$,  from Lemma~\ref{proofTheorem:Lemma:2}. From the foregoing, we have to bound the events complementary to that in \eqref{eq:armelim-casea1} for an arm $i$ to not get eliminated. Considering Chernoff-Hoeffding bound this is done as follows:
  \begin{align*}
&\mathbb{P}\left(\hat{r}_{i}\geq r_{i} + c_{m_i}\right)\leq \exp(-2c_{m_i}^{2}n_{m_i})\\
&\leq \exp(-2 * \frac{\rho_{a}\log (\psi T\epsilon_{m_{i}})}{2 n_{m_i}} *n_{m_i})
\leq \frac{1}{(\psi T\epsilon_{m_{i}})^{\rho_{a}}}   
  \end{align*}
Along similar lines, we have $\mathbb{P}\left(\hat{r}^{*}\leq r^{*} - c_{m_i}\right)\leq \frac{1}{(\psi  T\epsilon_{m_{i}})^{\rho_{a}}}.$ Thus, the probability that a sub-optimal arm ${i}$ is not eliminated in any round on or before $m_{i}$ is bounded above by $\bigg(\frac{2}{(\psi T\epsilon_{m_{i}})^{\rho_{a}}}\bigg)$. 
 Summing up over all arms in $A_{s^{*}}^{'}$ in conjunction with a simple bound of $T\Delta_{i}$ for each arm we obtain,
   \begin{align*}
&\sum_{i\in A_{s^{*}}^{'}}\left(\dfrac{2T\Delta_{i}}{(\psi T\epsilon_{m_{i}}^{2})^{\rho_{a}}}\right)
\leq\sum_{i\in A_{s^{*}}^{'}}\left(\frac{2T\Delta_{i}}{(\psi T\dfrac{\Delta_{i}^{4}}{16})^{\rho_{a}}}\right)
\overset{(a)}{\leq} \sum_{i\in A_{s^{*}}^{'}}\left(\frac{2T\Delta_{i}}{(\dfrac{T^2}{K^2}\dfrac{\Delta_{i}^{2}}{32})^{\frac{1}{2}}}\right)\leq 8\sqrt{2} \sum_{i\in A_{s^{*}}^{'}}K
%\leq \sum_{i\in A_{s^{*}}^{'}}\bigg(\frac{C_{1}(\rho_{a})T^{1-\rho_{a}}}{\Delta_{i}^{4\rho_{a}-1}}\bigg) %\text{, where } C_1(x) = \frac{2^{1+4x}}{\psi^{x}}
   \end{align*}
   
  Here, in $(a)$ we substituted the value $\rho_a$ and $\psi$.

%%%%%%%%%%%%%%%%%%%%%%%%%%%%%%%%%%%%%%%%%%%%%%%%%%%%%%%%%%%%%%%%%%%%%%%%%%%%%%%%%%%%%%%%%%%%%%%   



%%%%%%%%%%%%%%%%%%%%%%%%%%%%%%%%%%%%%%%%%%%%%%%%%%%%%%%%%%%%%%%%%%%%%%%%%%%%%%%%%%%%%%%%%%%%%%%   
\textbf{Case a2:} \textit{In round $\max(m_{i},g_{s_{k}})$, ${i} \in s_k$ for some $s_k \ne s^{*}$.}

Following a parallel argument like in Case $a1$, we have to bound the following two events of arm $a_{\max_{s_{k}}}$ not getting eliminated on or before $g_{s_{k}}$-th round,
\begin{align*}
  \hat{r}_{a_{\max_{s_{k}}}} \geq r_{a_{\max_{s_{k}}}} +c_{g_{s_{k}}} \text{ and } \hat{r}^{*} \leq r^{*} - c_{g_{s_{k}}} 
\end{align*} 

We can prove using Chernoff-Hoeffding bounds and considering independence of events mentioned above, that for $c_{g_{s_{k}}}=\sqrt{\frac{\rho_{s} \log (\psi T\epsilon_{g_{s_{k}}})}{2 n_{g_{s_{k}}}}}$ and  $n_{g_{s_{k}}}=\frac{\log{(\psi T\epsilon_{g_{s_{k}}}^{2})}}{2\epsilon_{g_{s_{k}}}}$ the probability of the above two events is bounded by $\bigg(\dfrac{2}{(\psi  T\epsilon_{g_{s_{k}}})^{\rho_{s}}}\bigg)$.

  Now, for any round $g_{s_{k}}$, all the elements of $C_{\max(m_{i},g_{s_{k}})}$ are the respective maximum payoff arms of their cluster $s_{k}, \forall s_{k}\in S$, and since clusters are fixed so we can bound the maximum probability that a sub-optimal arm ${i}\in A^{'}$  and ${i}\in s_{k}$ such that $a_{\max_{s_{k}}}\in C_{g_{s_{k}}}$ is not eliminated on or before the $g_{s_{k}}$-th round by the same probability as above. Summing up over all $p$ clusters and bounding the regret for each arm $i\in A_{s_{k}}^{'}$ trivially by $T\Delta_{i}$,
 \begin{align*}
 &\sum_{k=1}^{p}\sum_{i\in A_{s_{k}}^{'}}\left(\frac{2T\Delta_{i}}{(\psi T\frac{\Delta_{i}^{2}}{16})^{\rho_{s}}}\right) = \sum_{i\in A^{'}}\bigg( \frac{2T\Delta_{i}}{(\psi  T\frac{\Delta_{i}^{2}}{16})^{\rho_{s}}} \bigg) \\
 & \overset{(a)}{\leq} \sum_{i\in A^{'}}\left(\frac{2T\Delta_{i}}{(\frac{T^2}{K^2}\frac{\Delta_{i}^{2}}{32})^{\frac{1}{2}}}\right) = \sum_{i\in A^{'}} \left( 8\sqrt{2}K \right)
% &\leq \sum_{i\in A^{'}}\bigg(\frac{2^{1+4\rho_{s}}T^{1-\rho_{s}}}{\psi^{\rho_{s}}\Delta_{i}^{4\rho_{s}-1}}\bigg) = \sum_{i\in A^{'}}\frac{C_{1}(\rho_{s})T^{1-\rho_{s}}}{\Delta_{i}^{4\rho_{s}-1}}
 \end{align*}

Again we obtain $(a)$ by substituting the value of $\rho_s$ and $\psi$.

Summing the bounds in Cases $a1-a2$ and observing that the bounds in the aforementioned cases hold for any round $C_{\max \lbrace m_i,g_{s_k}\rbrace}$, we obtain the following contribution to the expected regret from case a:
   %Taking summation of the events mentioned above($a1$-$a4$) gives us an upper bound on the regret given that the optimal arm $a^{*}$ is still surviving, 
\begin{align*}
& \sum_{i\in A_{s^*}^{'}} 8\sqrt{2} K + \sum_{i\in A^{'}} 8\sqrt{2} K \leq \sum_{i\in A_{s^*}^{'}} 12 K + \sum_{i\in A^{'}} 12 K
%&\sum_{i\in A_{s^*}} \frac{C_{1}(\rho_{a})T^{1-\rho_{a}}}{\Delta_{i}^{4\rho_{a}-1}} + \sum_{i\in A^{'}}\bigg(\frac{C_{1}(\rho_{s})T^{1-\rho_{s}}}{\Delta_{i}^{4\rho_{s}-1}}\bigg)
\end{align*}

%%%%%%%%%%%%%%%%%%%%%%%%%%%%%%%%%%%%%%%%%%%%%%%%%%%%%%%%%%%%%%%%%%%%%%%%%%%%%%%%%%%%%%%%%%%%
\textbf{Case b:} \textit{For each arm $i$, either ${i}$ is eliminated in round $\max (m_{i},g_{s_{k}})$ or before or there is no optimal arm ${*}$ in $C_{\max(m_{i},g_{s_{k}})}$.} \\
\textbf{Case b1:} \textit{${*}\in C_{\max(m_{i},g_{s_{k}})}$ for each arm $i \in A'$ and cluster $s_k \in \check A$.} 
The condition in the case description above implies the following: \\
\begin{inparaenum}[\bfseries (i)]
\item each sub-optimal arm ${i}\in A^{'}$ is  eliminated on or before $\max (m_{i},g_{s_{k}})$ and hence  pulled not more than $ n_{m_{i}}$ number of times.\\
\item each sub-optimal cluster $s_k \in \check A$ is  eliminated on or before $\max (m_{i},g_{s_{k}})$ and hence  pulled not more than $ n_{g_{s_{k}}}$ number of times.
\end{inparaenum}

Hence, the maximum regret suffered due to pulling of a sub-optimal arm or a sub-optimal cluster is no more than the following:
 \begin{align*}
 &\sum_{i\in A^{'}}\Delta_{i}\bigg\lceil\dfrac{\log{(\psi T\epsilon_{m_{i}}^{2})}}{2\epsilon_{m_{i}}}\bigg\rceil 
\!+\! \sum_{k=1}^{p}\sum_{i\in A_{s_{k}}^{'}}\Delta_{i}\bigg\lceil\dfrac{\log{(\psi T\epsilon_{g_{s_{k}}}^{2})}}{2\epsilon_{g_{s_{k}}}}\bigg\rceil \\
%%%%%%%%%%%%%%%%%%%%
&\overset{a}{\leq}\sum_{i\in A^{'}}\Delta_{i}\bigg(1+\dfrac{16\log{\left(\psi T\left(\frac{\Delta_{i}}{2}\right)^{4}\right)}}{\Delta_{i}^{2}}\bigg) 
\quad+ \sum_{i\in A^{'}}\Delta_{i}\bigg(1+\dfrac{16\log{\left(\psi T\left(\frac{\Delta_{i}}{2}\right)^{4}\right)}}{\Delta_{i}^{2}}\bigg)
\\
%%%%%%%%%%%%%%%%%%%%
 &\overset{b}{\leq} \sum_{i\in A^{'}}\!\bigg[ 2\Delta_{i}+\dfrac{16(\log{(\frac{T^2}{K^2}\frac{\Delta_{i}^{4}}{1024})} + \log{(\frac{T^2}{K^2}\frac{\Delta_{i}^{4}}{1024})})}{\Delta_{i}} \bigg] \leq \sum_{i\in A^{'}}\!\bigg[ 2\Delta_{i}+\dfrac{32\left(\log{(\frac{T\Delta_{i}^2}{K})} + \log{(\frac{T\Delta_{i}^2}{K})}\right)}{\Delta_{i}} \bigg]
%  \\
% & \qquad \qquad +\dfrac{32\rho_{s}\log{(\psi T\dfrac{\Delta_{i}^{4}}{16\rho_{s}^{2}})}}{\Delta_{i}}\bigg\rbrace 
 \end{align*}
In the above, the $(a)$ follows since $\sqrt{2\epsilon_{m_{i}}} < \frac{\Delta_{i}}{4}$ and $\sqrt{2\epsilon_{n_{g_{s_{k}}}}} < \frac{\Delta_{a_{\max_{s_{k}}}}}{4}$ and $(b)$ is obtained by substituting the values of $\rho_a,\rho_s$ and $\psi$.

%&\leq\Delta_{i}\bigg(1+\dfrac{32\rho_{a}\log{(\psi T\dfrac{\Delta_{i}^{4}}{16\rho_{a}^{2}})}}{\Delta_{i}^{2}}\bigg)\\
%&\leq\Delta_{i}\bigg\lceil\dfrac{2\log{(\psi T(\dfrac{\Delta_{i}}{2\sqrt{\rho_{a}})^{4})}}}{(\dfrac{\Delta_{i}}{2\sqrt{\rho_{a}}})^{2}}\bigg\rceil \\
%\text{, since } \sqrt{\rho_{a}\epsilon_{m_{i}}}\leq\dfrac{\Delta_{i}}{2}
 
%%%%%%%%%%%%%%%%%%%%%%%%%%%%%%%%%%%%%%%%%%%%%%%%%%%%%%%%%%%%%%%%%%%%%%%%%%%%%%%%%%%%%%%%%%%%%%%   
%\textbf{Case b2:} \textit{Optimal arm $a^{*}$ is eliminated by a sub-optimal arm.}\\
  %
	%This, can happen in $3$ ways,
%\newline
\textbf{Case b2:} \textit{${*}$ is eliminated by some sub-optimal arm in $s^*$} \\
%In this case, we are concerned with the arm elimination condition only. 
Optimal arm $*$ can get eliminated by some sub-optimal arm $i$ only if arm elimination condition holds, i.e., 
\begin{align*}
\hat r_{i} - c_{m_i} > \hat{r}^{*}+ c_{m_i},
\end{align*}
where, as mentioned before, $c_{m_i}  =\sqrt{\frac{\rho_{a}\log (\psi T\epsilon_{m_{i}})}{2 n_{m_i}}}$.
From analysis in Case $a1$, notice that, if \eqref{eq:armelim-casea1} holds in conjunction with the above, arm $i$ gets eliminated. Also, recall from Case $a1$ that the events complementary to \eqref{eq:armelim-casea1} have low-probability and can be upper bounded by $\frac{2}{(\psi  T\epsilon_{m_{*}})^{\rho_{a}}}$. Moreover, a sub-optimal arm that eliminates $*$ has to survive until round $m_*$. In other words, all arms ${j}\in s^{*}$ such that $m_{j} < m_{*}$ are eliminated on or before $m_*$ (this corresponds to case b1). Let, the arms surviving till $m_{*}$ round be denoted by $A^{'}_{s^{*}}$. This leaves any arm $a_{b}$ such that $m_{b}\geq m_{*} $ to still survive and eliminate arm ${*}$ in round $m_{*}$. Let, such arms that survive ${*}$ belong to $A^{''}_{s^{*}}$. Also maximal regret per step after eliminating ${*}$ is the maximal $\Delta_{j}$ among the remaining arms in $A^{''}_{s^{*}}$ with $m_{j}\geq m_{*}$.  Let $m_{b}=\min\lbrace m|\sqrt{2\epsilon_{m}}<\frac{\Delta_{b}}{4}\rbrace$. Hence, the maximal regret after eliminating the arm ${*}$ is upper bounded by, 
%Let $C_2(x) = \frac{2^{2x+\frac{3}{2}}}{\psi^{x}}$.
\begin{align*}
&\sum_{m_{*}=0}^{max_{j\in A^{'}_{s^{*}}}m_{j}}\sum_{\substack{i\in A^{''}_{s^{*}}: \\ m_{i}\geq m_{*}}}\bigg(\dfrac{2}{(\psi  T\epsilon_{m_{*}})^{\rho_{a}}} \bigg).T\max_{\substack{j\in A^{''}_{s^{*}}: \\ m_{j}\geq m_{*}}}{\Delta}_{j}\\
%%%%%%%%%%%%%%%
&\leq\sum_{m_{*}=0}^{max_{j\in A^{'}_{s^{*}}}m_{j}}\sum_{i\in A^{''}_{s^{*}}:m_{i} \geq m_{*}}\bigg(\dfrac{2}{(\psi  T\epsilon_{m_{*}})^{\rho_{a}}} \bigg).T.4\sqrt{2\epsilon_{m_{*}}} \\
%%%%%%%%%%%%%%%
&\leq\sum_{m_{*}=0}^{max_{j\in A^{'}_{s^{*}}}m_{j}}\sum_{i\in A^{''}_{s^{*}}:m_{i} \geq m_{*}}8\sqrt{2}\bigg(\dfrac{T^{1-\rho_{a}}}{\psi^{\rho_{a}}\epsilon_{m_{*}}^{\rho_{a}-\frac{1}{2}}} \bigg)\\
%%%%%%%%%%%%%%%
&\leq\sum_{i\in A^{''}_{s^{*}}:m_{i} \geq m_{*}}\sum_{m_{*}=0}^{\min{\lbrace m_{i},m_{b}\rbrace}}\bigg(\dfrac{8\sqrt{2} T^{1-\rho_{a}}}{\psi^{\rho_{a}}2^{-(\rho_{a}-\frac{1}{2})m_{*}}} \bigg)\\
%%%%%%%%%%%%%%
&\!\leq\!\!\sum_{i\in A^{'}_{s^{*}}}\frac{8\sqrt{2} T^{1-\rho_{a}}}{\psi^{\rho_{a}}2^{-(\rho_{a}-\frac{1}{2})m_{*}}}\! +\!\!\!\sum_{i\in A^{''}_{s^{*}}\setminus A^{'}_{s^{*}}}\!\frac{8\sqrt{2} T^{1-\rho_{a}}}{\psi^{\rho_{a}}2^{-(\rho_{a}-\frac{1}{2})m_{b}}} \\
%%%%%%%%%%%%%%
&\!\leq\!\!\sum_{i\in A^{'}_{s^{*}}}\frac{T^{1-\rho_{a}}2^{\rho_{a}+\frac{7}{2}}}{\psi^{\rho_{a}}\Delta_{i}^{2\rho_{a}-1}} \!+\!\!\!\sum_{i\in A^{''}_{s^{*}}\setminus A^{'}_{s^{*}}}\!\!\frac{T^{1-\rho_{a}}2^{\rho_{a}+\frac{7}{2}}}{\psi^{\rho_{a}}b^{2\rho_{a}-1}} \\
%%%%%%%%%%%%%%
& \leq \sum_{i\in A^{'}_{s^{*}}} 16K \!+\!\!\!\sum_{i\in A^{''}_{s^{*}}\setminus A^{'}_{s^{*}}}\!\! 16K
%& = \sum_{i\in A^{'}_{s^{*}}}\dfrac{ C_{2}(\rho_{a}) T^{1-\rho_{a}}}{\Delta_{i}^{4\rho_{a}-1}} +\sum_{i\in A^{''}_{s^{*}}\setminus A^{'}_{s^{*}}}\dfrac{C_{2(\rho_{a})}T^{1-\rho_{a}}}{b^{4\rho_{a}-1}}.
\end{align*}

%%%%%%%%%%%%%%%%%%%%%%%%%%%%%%%%%%%%%%%%%%%%%%%%%%%%%%%%%%%%%%%%%%%%%%%%%%%%%%%%%%%%%%%%%%%%%%%

%%%%%%%%%%%%%%%%%%%%%%%%%%%%%%%%%%%%%%%%%%%%%%%%%%%%%%%%%%%%%%%%%%%%%%%%%%%%%%%%%%%%%%%%%%%%%%%   
\textbf{Case b3:} \textit{$s^{*}$ is eliminated by some sub-optimal cluster.} 
Let $C_{g}^{'}=\lbrace a_{max_{s_{k}}}\in A^{'}|\forall s_{k}\in S \rbrace$ and $C_{g}^{''}=\lbrace a_{max_{s_{k}}}\in A^{''}|\forall s_{k}\in S \rbrace$. A sub-optimal cluster $s_k$ will eliminate $s^*$ in round $g_*$ only if the cluster elimination condition of Algorithm \ref{alg:clusucb} holds, which is the following when ${*}\in C_{g_{*}}$:
\begin{align}
\hat r_{a_{\max_{s_k}}} - c_{g_*} > \hat{r}^{*}+ c_{g_*}.
\label{eq:caseb3-cluselim}
\end{align}
Notice that when ${*}\notin C_{g_{*}}$, since $r_{a_{max_{s_{k}}}}>r^{*}$, the inequality in \eqref{eq:caseb3-cluselim} has to hold for cluster $s_k$ to eliminate $s^*$.
As in case $b2$, the probability that a given sub-optimal cluster $s_k$ eliminates $s^*$ is upper bounded by  $\frac{2}{(\psi T\epsilon_{g_{s^{*}}})^{\rho_{s}}}$ and all sub-optimal clusters with $g_{s_{j}}< g_{*}$ are eliminated before round $g_*$. This leaves any arm $a_{\max_{s_{b}}}$ such that $g_{s_{b}}\geq g_{*}$ to still survive and eliminate arm ${*}$ in round $g_{*}$. Let, such arms that survive ${*}$ belong to $C_{g}^{''}$. Hence, following the same way as case $b2$,  the maximal regret after eliminating ${*}$ is,
 \begin{align*}
 \!\!\sum_{g_{*}=0}^{\max\limits_{a_{\max_{s_{j}}}\in C_{g}^{'}}g_{s_{j}}}\!\!\!\!\!\sum_{\substack{\scriptsize a_{\max_{s_{k}}}\in C_{g}^{''}: \\ g_{s_{k}} \geq g_{*}}}\bigg(\dfrac{2}{(\psi T\epsilon_{g_{s^{*}}})^{\rho_{s}}} \bigg)T\max_{\substack{a_{\max_{s_{j}}}\in C_{g}^{''}: \\ g_{s_{j}}\geq g_{*}}}{\Delta}_{a_{\max_{s_{j}}}}
 \end{align*}
Using $A'\supset C_{g}^{'}$ and $A''\supset C_{g}^{''}$, we can bound the regret contribution from this case in a similar manner as Case $b2$ as follows:
\begin{align*}
 &\!\!\sum_{i\in A^{'}\setminus A_{s^*}^{'}}\frac{T^{1-\rho_{s}}2^{\rho_{s}+\frac{5}{2}}}{\psi^{\rho_{s}}\Delta_{i}^{2\rho_{s}-1}} 
 \!+\!\!\!\sum_{i\in A^{''}\setminus A^{'}\cup A_{s^*}^{'}}\!\!\!\!\frac{T^{1-\rho_{s}}2^{\rho_{s}+\frac{5}{2}}}{\psi^{\rho_{s}}b^{2\rho_{s}-1}} \\
 & = \sum_{i\in A^{'}\setminus A_{s^*}^{'}} 16K +\sum_{i\in A^{''}\setminus A^{'}\cup A_{s^*}^{'}} 16K
% & = \sum_{i\in A^{'}\setminus A_{s^*}^{'}}\frac{C_{2}(\rho_{s})T^{1-\rho_{s}}}{\Delta_{i}^{4\rho_{s}-1}} +\sum_{i\in A^{''}\setminus A^{'}\cup A_{s^*}^{'}}\frac{C_{2}(\rho_{s})T^{1-\rho_{s}}}{b^{4\rho_{s}-1}} 
\end{align*}

%%%%%%%%%%%%%%%%%%%%%%%%%%%%%%%%%%%%%%%%%%%%%%%%%%%%%%%%%%%%%%%%%%%%%%%%%%%%%%%%%%%%%%%%%
\textbf{Case b4:} \textit{${*}$ is not in $C_{\max(m_{i},g_{s_{k}})}$, but belongs to $B_{\max(m_{i},g_{s_{k}})}$.}

In this case the optimal arm ${*}\in s^{*}$ is not eliminated, also $s^{*}$ is not eliminated. So, for all sub-optimal arms $i$ in $A_{s^*}^{'}$ which gets eliminated on or before $\max \lbrace m_{i},g_{s_{k}} \rbrace$ will get pulled no more than $ \left\lceil\dfrac{\log{(\psi T\epsilon_{m_{i}}^{2})}}{2\epsilon_{m_{i}}}\right\rceil$ number of times, which leads to the following bound the contribution to the expected regret, as in Case $b1$:
\begin{align*}
 &\sum_{i\in A_{s^*}^{'}}\bigg\lbrace \Delta_{i}+\dfrac{32\log{(\frac{T\Delta_i^2}{K})}}{\Delta_{i}} \bigg\rbrace 
\end{align*} 

For arms $a_i \notin s^*$, the contribution to the regret cannot be greater than that in Case $b3$. So the regret is bounded by,

\begin{align*}
\sum_{i\in A^{'}\setminus A_{s^*}^{'}} 16K +\sum_{i\in A^{''}\setminus A^{'} \cup A_{s^*}^{'}} 16K
%\sum_{i\in A^{'}\setminus A_{s^*}^{'}}\dfrac{C_{2}(\rho_{s})T^{1-\rho_{s}}}{\Delta_{i}^{4\rho_{s}-1}} +\sum_{i\in A^{''}\setminus A^{'} \cup A_{s^*}^{'}}\dfrac{C_{2}(\rho_{s})T^{1-\rho_{s}}}{b^{4\rho_{s}-1}}
\end{align*}
The main claim follows by summing the contributions to the expected regret from each of the cases above.
\end{proof}





\section{Proof of Proposition 1}
\label{App:A}

\begin{proof}
Let $p=1$ such that all the arms in $A$ belongs to a single cluster. Hence, in ClusUCB-AE there is only arm elimination and no cluster elimination. Let, for each sub-optimal arm ${i}$, $m_{i}=\min{\lbrace m|\sqrt{\epsilon_{m}} < \dfrac{\Delta_{i}}{2} \rbrace}$. Also $\rho_{a}=\frac{1}{2}$ is a constant in this proof. Let $A^{'}=\lbrace i\in A: \Delta_{i} > b \rbrace$ and $A^{''}=\lbrace i\in A: \Delta_{i} > 0 \rbrace$. 

%Also $z_{i}$ denotes total number of times an arm $i$ has been pulled. In the $m$-th round, $n_{m}$ denotes the number of pulls allocated to the surviving arms in $B_{m}$. 
%The theoretical analysis remains same as we have always bounded the values of $\rho_{a}\in (0,1]$.

\subsection*{Case $a$: \textit{Some sub-optimal arm ${i}$ is not eliminated in round $m_{i}$ or before and the optimal arm ${*}\in B_{m_{i}}$}}
  
	Following the steps of Theorem \ref{Result:Theorem:1} Case $a1$, an arbitrary sub-optimal arm ${i}\in A^{'}$ can get eliminated only when the event,
	\begin{align}
	\hat{r}_{i}  \le r_{i} + c_{m_i} \text{ and } \label{eq:appA:armelim-casea}
 	\hat{r}^{*}\geq  r^{*} - c_{m_i}
	\end{align}
	
	takes place. So to bound the regret we need to bound the probability of the complementary event of these two conditions. Note that  $c_{m_{i}} = \sqrt{\frac{\rho_{a}\log (\psi T\epsilon_{m_{i}})}{2 n_{m_i}}}$. A sub-optimal arm $i$ will get eliminated in the $m_i$-th round because $n_{m_{i}}=\dfrac{\log{(\psi T\epsilon_{m_{i}}^{2})}}{2\epsilon_{m_{i}}}$ and substituting this in $c_{m_i}$ and applying Lemma \ref{proofTheorem:Lemma:2} we get, $c_{m_i} < \dfrac{\Delta_{i}}{4} $.

  Again, for ${i} \in A^{'}$, 
  \begin{align*}
\hat{r}_{i} + c_{m_i}&\leq r_{i} + 2c_{m_i} 
%&= \hat{r}_{i} + 4c_{m_{i}} - 2c_{m_{i}} \\
 < r_{i} + \Delta_{i} - 2c_{m_i}
 \leq r^{*} -2c_{m_i} 
 \leq \hat{r}^{*} - c_{m_i}
  \end{align*}

	Applying Chernoff-Hoeffding bound and considering independence of complementary of the two events in \ref{eq:appA:armelim-casea},
  \begin{align*}
\mathbb{P}\lbrace\hat{r}_{i}&\geq r_{i} + c_{m_i}\rbrace\leq \exp(-2c_{m_i}^{2}n_{m_{i}})
\leq \exp(-2 * \dfrac{\rho_{a}\log (\psi T\epsilon_{m_{i}})}{2 n_{m_{i}}} *n_{m_{i}})
\leq \dfrac{1}{(\psi T\epsilon_{m_{i}})^{\rho_{a}}}   
  \end{align*}
 
%$\leq \bigg(\dfrac{1}{4\psi T\epsilon_{m}^{2}}\bigg)^{D}$, as $\ell_{m}-1\leq D$
% \hspace*{2em}
 
Similarly, $\mathbb{P}\lbrace\hat{r}^{*}\leq r^{*} - c_{m_i}\rbrace\leq \dfrac{1}{(\psi  T\epsilon_{m_{i}})^{\rho_{a}}}$. Summing the two up, the probability that a sub-optimal arm ${i}$ is not eliminated on or before $m_{i}$-th round is  $\bigg(\dfrac{2}{(\psi T\epsilon_{m_{i}})^{\rho_{a}}}\bigg)$. 
 
Summing up over all arms in $A^{'}$ and bounding the regret for each arm $i\in A^{'}$ trivially by $T\Delta_{i}$, we obtain
   \begin{align*}
\sum_{i\in A^{'}}\bigg(\dfrac{2T\Delta_{i}}{(\psi T\epsilon_{m_{i}})^{\rho_{a}}}\bigg)
\leq\sum_{i\in A^{'}}\bigg(\dfrac{2T\Delta_{i}}{(\psi T\dfrac{\Delta_{i}^{2}}{32})^{\rho_{a}}}\bigg)
&\leq \sum_{i\in A^{'}}\bigg(\dfrac{2^{1+4\rho_{a}}T^{1-\rho_{a}}\Delta_{i}}{\psi^{\rho_{a}}\Delta_{i}^{2\rho_{a}}}\bigg)
\leq \sum_{i\in A^{'}}\bigg(\dfrac{2^{1+5\rho_{a}}T^{1-\rho_{a}}}{\psi^{\rho_{a}}\Delta_{i}^{2\rho_{a}-1}}\bigg)\\  
%%%%%%%%%%%%%%%%%% 
& \overset{(a)}{\leq}\sum_{i\in A^{'}}\leq 8\sqrt{2} K
   \end{align*}

Here, $(a)$ is obtained by substituting the values of $\psi$ and $\rho_a$.

\subsection*{Case $b$: \textit{Either an arm ${i}$ is eliminated in round $m_{i}$ or before or else there is no optimal arm ${*}\in B_{m_{i}}$ }}

\subsubsection*{Case $b1$: \textit{${*}\in B_{m_{i}}$ and each ${i}\in A^{'}$ is  eliminated on or before $m_{i}$ } }

 Since we are eliminating a sub-optimal arm ${i}$ on or before round $m_{i}$, it is pulled no longer than, 
 \begin{align*}
  \bigg\lceil\dfrac{\log{(\psi T\epsilon_{m_{i}}^{2})}}{2\epsilon_{m_{i}}}\bigg\rceil
 \end{align*}

So, the total contribution of ${i}$  till round $m_{i}$ is given by, 
\begin{align*}
&\Delta_{i}\bigg\lceil\dfrac{\log{(\psi T\epsilon_{m_{i}}^{2})}}{2\epsilon_{m_{i}}}\bigg\rceil
\leq\Delta_{i}\bigg\lceil\dfrac{\log{(\psi T(\dfrac{\Delta_{i}}{4\sqrt{2}})^{4})}}{(\dfrac{\Delta_{i}}{4\sqrt{2}})^{2}}\bigg\rceil \text{, since } \sqrt{2\epsilon_{m_{i}}} < \dfrac{\Delta_{i}}{4}\\
&\overset{(a)}{\leq}\Delta_{i}\bigg(1+\dfrac{32\log{(\frac{T}{K^2} T(\Delta_{i})^{4})}}{\Delta_{i}^{2}}\bigg)
\leq\Delta_{i}\bigg(1+\dfrac{32\log{( \frac{T\Delta_i^2}{K})}}{\Delta_{i}^{2}}\bigg)
\end{align*} 
 
In the above case, $(a)$ is obtained by substituting the values of $\psi$ and $\rho_a$. Summing over all arms in $A^{'}$ the total regret is given by, 
\begin{align*}
\sum_{i\in A^{'}}\Delta_{i}\bigg(1+\dfrac{32\log{( \frac{T\Delta_i^2}{K})}}{\Delta_{i}^{2}}\bigg)
\end{align*}

\subsubsection*{Case $b2$: \textit{Optimal arm ${*}$ is eliminated by a sub-optimal arm  }}


	Firstly, if conditions of Case $a$ holds then the optimal arm ${*}$ will not be eliminated in round $m=m_{*}$ or it will lead to the contradiction that $r_{i}>r^{*}$. In any round $m_{*}$, if the optimal arm ${*}$ gets eliminated then for any round from $1$ to $m_{j}$ all arms ${j}$ such that $m_{j}< m_{*}$ were eliminated according to assumption in Case $a$. Let the arms surviving till $m_{*}$ round be denoted by $A^{'}$. This leaves any arm $a_{b}$ such that $m_{b}\geq m_{*}$ to still survive and eliminate arm ${*}$ in round $m_{*}$. Let such arms that survive ${*}$ belong to $A^{''}$. Also maximal regret per step after eliminating ${*}$ is the maximal $\Delta_{j}$ among the remaining arms ${j}$ with $m_{j}\geq m_{*}$.  Let $m_{b}=\min\lbrace m|\sqrt{2\epsilon_{m}}<\dfrac{\Delta_{b}}{4}\rbrace$. Hence, the maximal regret after eliminating the arm ${*}$ is upper bounded by, 
\begin{align*}
&\sum_{m_{*}=0}^{max_{j\in A^{'}}m_{j}}\sum_{i\in A^{''}:m_{i}>m_{*}}\bigg(\dfrac{2}{(\psi  T\epsilon_{m_{*}})^{\rho_{a}}} \bigg).T\max_{j\in A^{''}:m_{j}\geq m_{*}}{\Delta}_{j}\\
%%%%%%%%%%%%%%%%%%%%%%%%%%%
&\leq\sum_{m_{*}=0}^{max_{j\in A^{'}}m_{j}}\sum_{i\in A^{''}:m_{i}>m_{*}}\bigg(\dfrac{2}{(\psi  T\epsilon_{m_{*}})^{\rho_{a}}} \bigg).T.4\sqrt{2}\sqrt{\epsilon_{m_{*}}}\\
%%%%%%%%%%%%%%%%%%%%%%%%%%
&\leq\sum_{m_{*}=0}^{max_{j\in A^{'}}m_{j}}\sum_{i\in A^{''}:m_{i}>m_{*}}8\sqrt{2}\bigg(\dfrac{T^{1-\rho_{a}}}{\psi^{\rho_{a}}\epsilon_{m_{*}}^{\rho_{a}-\frac{1}{2}}} \bigg)\\
%%%%%%%%%%%%%%%%%%%%%%%%%%
&\leq\sum_{i\in A^{''}:m_{i}>m_{*}}\sum_{m_{*}=0}^{\min{\lbrace m_{i},m_{b}\rbrace}}\bigg(\dfrac{8\sqrt{2}T^{1-\rho_{a}}}{\psi^{\rho_{a}}2^{-(\rho_{a}-\frac{1}{2})m_{*}}} \bigg)\\
%%%%%%%%%%%%%%%%%%%%%%%%%%
&\leq\sum_{i\in A^{'}}\bigg(\dfrac{8\sqrt{2}T^{1-\rho_{a}}}{\psi^{\rho_{a}}2^{-(\rho_{a}-\frac{1}{2})m_{*}}} \bigg)+\sum_{i\in A^{''}\setminus A^{'}}\bigg(\dfrac{8\sqrt{2}T^{1-\rho_{a}}}{\psi^{\rho_{a}}2^{-(\rho_{a}-\frac{1}{2})m_{b}}} \bigg)\\
%%%%%%%%%%%%%%%%%%%%%%%%%%
&\leq\sum_{i\in A^{'}}\bigg(\dfrac{4T^{1-\rho_{a}}*2^{\rho_{a}-\frac{1}{2}}}{\psi^{\rho_{a}}\Delta_{i}^{8\sqrt{2}\rho_{a}-1}} \bigg)+\sum_{i\in A^{''}\setminus A^{'}}\bigg(\dfrac{8\sqrt{2}T^{1-\rho_{a}}*2^{\rho_{a}-\frac{1}{2}}}{\psi^{\rho_{a}}b^{2\rho_{a}-1}} \bigg)\\
%%%%%%%%%%%%%%%%%%%%%%%%%%
&\leq\sum_{i\in A^{'}}\bigg(\dfrac{T^{1-\rho_{a}}2^{\rho_{a}+\frac{7}{2}}}{\psi^{\rho_{a}}\Delta_{i}^{2\rho_{a}-1}} \bigg)+\sum_{i\in A^{''}\setminus A^{'}}\bigg(\dfrac{T^{1-\rho_{a}}2^{\rho_{a}+\frac{7}{2}}}{\psi^{\rho_{a}}b^{2\rho_{a}-1}} \bigg)\\
%%%%%%%%%%%%%%%%%%%%%%%%%%
&\overset{(a)}{\leq}\sum_{i\in A^{'}}16K +\sum_{i\in A^{''}\setminus A^{'}} 16K\\
%& = \sum_{i\in A^{'}}\bigg(\dfrac{ C_{2}(\rho_{a}) T^{1-\rho_{a}}}{\Delta_{i}^{4\rho_{a}-1}} \bigg)+\sum_{i\in A^{''}\setminus A^{'}}\bigg(\dfrac{C_{2(\rho_{a})}T^{1-\rho_{a}}}{b^{4\rho_{a}-1}} \bigg) \text{, where } C_2(x) = \frac{2^{2x+\frac{3}{2}}}{\psi^{x}}
\end{align*}

Again $(a)$ is obtained by substituting the values of $\psi$ and $\rho_a$. Summing up \textbf{Case a} and \textbf{Case b}, the total regret till round $m$ is given by,
\begin{align*}
 \E[R_{T}] \leq &\sum\limits_{i\in A:\Delta_{i} > b} \left\lbrace 12K + \bigg(\Delta_{i}+\dfrac{32\log{(\frac{T\Delta_i^2}{K})}}{\Delta_{i}}\bigg) + 16K\right\rbrace +\sum\limits_{i\in A:0 < \Delta_{i}\leq b} 16K + \max_{i\in A:\Delta_{i}\leq b}\Delta_{i}T
\end{align*}
\end{proof}


\section{Proof of Corollary 1}
\label{App:Proof:Corollary:1}
\begin{proof}
%As stated in \cite{auer2010ucb}, the regret bound can be of the order of $\sqrt{KT\log K}$ in non-stochastic MAB setting. This is shown in Exp4\cite{auer2002nonstochastic} algorithm. 
First we recall the definition of Theorem \ref{Result:Theorem:1} below,
\begin{align*}
&\E [R_{T}]\leq 
\sum\limits_{\substack{i\in A_{s^{*}},\\\Delta_{i} > b}}\bigg\lbrace \Delta_{i} + 12K
+ \frac{32\log{(\frac{T\Delta_i^2}{K})}}{\Delta_{i}} \bigg\rbrace
 + \! \! \sum\limits_{\substack{i\in A,\\\Delta_{i} > b}} \bigg\lbrace 2\Delta_{i} +
12K + \frac{64\log{(\frac{T\Delta_i^2}{K})}}{\Delta_{i}} \bigg\rbrace \\
%%%%%%%%%%%%%%%%%
&+ \sum\limits_{\substack{i\in A_{s^{*}},\\ \Delta_{i} > b}} 
16K+\sum\limits_{\substack{i\in A_{s^{*}},\\0 < \Delta_{i}\leq b}} 16K + \sum_{\substack{i\in A\setminus A_{s^*}:\\\Delta_{i}> b}}32K +\sum_{\substack{i\in A \setminus A_{s^*}:\\ 0 < \Delta_{i} \leq b}}32K 
 \!+\! \max\limits_{i:\Delta_{i}\leq b}\Delta_{i}T
\end{align*}

Now we know from \cite{bubeck2011pure} that the function $x\in [0,1]\mapsto x\exp(-Cx^2)$ is  decreasing on $\left[\dfrac{1}{\sqrt{2C}},1\right ]$ for any $C>0$. So, taking $C=\left\lfloor \dfrac{T}{e}\right\rfloor$ and by choosing  $\Delta_{i}=\Delta=\sqrt{\dfrac{K\log K}{T}}>\sqrt{\dfrac{e}{T}}$ for all ${i:i\neq *}\in A$ and substituting $p=\left\lceil \dfrac{K}{\log K}\right\rceil $ in the bound of ClusUCB we get,

	\begin{align*}
	\sum_{i\in A_{s^{*}}:\Delta_{i} > b} 12K =12\dfrac{K^2}{p}
	\end{align*}		
	 Similarly, for the term, 
	 \begin{align*}
	 \sum_{i\in A:\Delta_{i} > b} 12K = 12 K^2
	 \end{align*}
	 
	
	For the term regarding number of pulls,
	\begin{align*}
	\sum_{i\in A:\Delta_{i} > b}\dfrac{64\log{(\frac{T\Delta_i^2}{K})}}{\Delta_{i}} &\leq  \dfrac{64K\sqrt{T}\log{(T\dfrac{K\log K}{T K})}}{\sqrt{K\log K}} \leq  \dfrac{64\sqrt{KT}\log{(\log K)}}{\sqrt{\log K}}\\
	%%%%%%%%%%%%%%%%%%%%%%%
	&\overset{(a)}{\leq} 64\sqrt{KT}
	\end{align*}		
	
	Here $(a)$ is obtained by the identity $\dfrac{\log\log K}{\sqrt{\log K}} < 1$ for $K\geq 2$. Lastly we can bound the error terms as, 
	\begin{align*}
	\sum\limits_{i\in A_{s^{*}}:0\leq\Delta_{i}\leq b} 16K =\dfrac{16K^2}{p} \overset{<}{(a)} 16K\log K
	\end{align*}	 	
 	Here we obtain $(a)$ by substituting the value of $p$. Similarly for the term,
 	\begin{align*}
 	\sum_{i\in A\setminus A_{s^*}: \Delta_{i} > b} 16K =\dfrac{16K^2}{p} < 16K\log K
	\end{align*} 	
	Also, for all $b\geq \sqrt{\dfrac{e}{T}}$,
	\begin{align*}
 	\sum_{i\in A\setminus A_{s^*}: 0 < \Delta_{i} \leq b} 32K = \left(K-\dfrac{K}{p}\right) 32K
	\end{align*} 	
	
	Now, $K-\dfrac{K}{p}= K\left( \dfrac{p-1}{p} \right) < K\left(  \dfrac{\frac{K}{\log K}+1-1}{\frac{K}{\log K}+1 }\right) < \dfrac{K^2}{K+\log K}$. So, after substituting the value of $p=\left\lceil \dfrac{K}{\log K} \right\rceil$, we get,
	
	\begin{align*}
 	\sum_{i\in A\setminus A_{s^*}: 0 < \Delta_{i} \leq b} 32K = \left(K-\dfrac{K}{p}\right)32K < \dfrac{32 K^3}{K+\log K}
	\end{align*} 	
	
	Summing up all the contribution from the individual cases as shown above, the total gap-independent regret is given by,	
	
	\begin{align*}
	\E[R_{T}]\leq & 12K\log K + 32\sqrt{KT} + 12K^2 + 64\sqrt{KT} + 32K\log K  \dfrac{64 K^3}{K+\log K}
	\end{align*}
 	
	So, the total bound for using both arm and cluster elimination cannot be worse than,
	
	\begin{align*}
	\E[R_{T}]\leq 96\sqrt{KT} + 12K^2 + 44K\log K + \dfrac{64 K^3}{K+\log K}\\ 
	\end{align*}		
\end{proof}

%\section{Why Clustering?}
%\label{App:E}
%
%In this section we want to specify the apparent use of clustering. The error bounds are shown in Table \ref{App:E:table:3}.
%
%\begin{table}[!h]
%\caption{Error Bound}
%\label{App:E:table:3}
%\begin{center}
%\begin{tabular}{p{1.4cm}p{10.3cm}p{3.5cm}}
%\multicolumn{1}{c}{\bf Elim Type} &\multicolumn{1}{c}{\bf Error Bound} &\multicolumn{1}{c}{\bf Remarks} \\
%\hline \\
%Only Arm Elimination (ClusUCB-AE)	& \begin{align*}\underbrace{\sum_{i\in A:\Delta_{i} > b}\bigg(\dfrac{C_{2}(\rho_{a})T^{1-\rho_{a}}}{\Delta_{i}^{4\rho_{a} -1}} \bigg)}_{\text{Case b2, Proposition \ref{proofTheorem:Prop:1}}} + \underbrace{\sum_{i\in A:0 < \Delta_{i}\leq b}\bigg( \dfrac{C_{2}(\rho_{a})T^{1-\rho_{a}}}{b^{4\rho_{a} -1}} \bigg)}_{\text{Case b2, Proposition \ref{proofTheorem:Prop:1}}}\end{align*}  & With $\rho_{a}=\frac{1}{2},$ and $\psi=\frac{T}{196 \log K}$ this gives $300\sqrt{KT}+300\sqrt{KT\log K}$. Hence, this has an order of $O(\sqrt{KT\log K})$.\\
%\hline\\
%%%%%%%%%%%%%%%%%%%%%%%%%%%%%%%%%%%%%%%%%%%%%%%%%%%%%%%%%%%%%%%%%%%%%%%%%%%
%%%%%%%%%%%%%%%%%%%%%%%%%%%%%%%%%%%%%%%%%%%%%%%%%%%%%%%%%%%%%%%%%%%%%%%%%%%
%Arm \& Cluster Elimination (ClusUCB) 	& \begin{align*}  \underbrace{\sum_{i\in A_{s^{*}}:\Delta_{i} > b}\bigg(\dfrac{C_{2}(\rho_{a})T^{1-\rho_{a}}}{\Delta_{i}^{4\rho_{a}-1}} \bigg)+ \sum_{i\in A_{s^{*}}:0\leq\Delta_{i}\leq b}\bigg(\dfrac{C_{2}(\rho_{a})T^{1-\rho_{a}}}{b^{4\rho_{a} -1}} \bigg)}_{\text{Case b2, Arm Elim, Theorem \ref{Result:Theorem:1}}}\\   
% + \underbrace{\sum_{i\in A\setminus A_{s^*}:\Delta_{i} > b}\bigg(\dfrac{2C_{2}(\rho_{s})T^{1-\rho_{s}}}{\Delta_{i}^{4\rho_{s}-1}} \bigg)+ \sum_{i\in A\setminus A_{s^*}:0\leq\Delta_{i}\leq b}\bigg(\dfrac{2C_{2}(\rho_{s})T^{1-\rho_{s}}}{b^{4\rho_{s} -1}} \bigg)}_{\text{Case b3+b4, Clus Elim, Theorem \ref{Result:Theorem:1}}} \end{align*} & With $\rho_{a}=\frac{1}{2}$, $\rho_{s}=\frac{1}{2}, p=\lceil \frac{K}{\log K}\rceil$ and $\psi=\frac{T}{196 \log K}$ this gives $\frac{300 \sqrt{T}\log K^{\frac{3}{2}} }{\sqrt{K}} + \frac{300 \sqrt{T}\log K}{\sqrt{K}} + 600 \frac{K}{K+\log K}\sqrt{KT\log K} + 600 \frac{K}{K+\log K}\sqrt{KT}$. So we can reduce the error bound to $O(\frac{K}{K+\log K}\sqrt{KT\log K})$.\\
%\hline
%\end{tabular}
%\end{center}	
%\end{table}
%
%While looking at the error terms in Table~\ref{App:E:table:3}, we see that using just arm elimination (ClusUCB-AE) the elimination error bound is more than using both arm and cluster  elimination simultaneously (ClusUCB). 

