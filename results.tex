	
We now state the main result that upper bounds the expected regret of ClusUCB.
\begin{theorem}[\textbf{\textit{Gap dependent regret bound}}]
\label{Result:Theorem:1}
For $T\geq K^{2.4} $, $\rho_a =\frac{1}{2}$, $\rho_s =\frac{1}{2}$ and $\psi=\frac{T}{K^2}$ the regret $R_T$ of ClusUCB satisfies
\begin{align*}
&\E [R_{T}]\leq 
\sum\limits_{\substack{i\in A_{s^{*}},\\\Delta_{i} > b}}\bigg\lbrace \Delta_{i} + 12K
+ \frac{32\log{(\frac{T\Delta_i^2}{K})}}{\Delta_{i}} \bigg\rbrace
 + \! \! \sum\limits_{\substack{i\in A,\\\Delta_{i} > b}} \bigg\lbrace 2\Delta_{i} +
12K + \frac{64\log{(\frac{T\Delta_i^2}{K})}}{\Delta_{i}} \bigg\rbrace \\
%%%%%%%%%%%%%%%%%
&+ \sum\limits_{\substack{i\in A_{s^{*}},\\ \Delta_{i} > b}} 
16K+\sum\limits_{\substack{i\in A_{s^{*}},\\0 < \Delta_{i}\leq b}} 16K + \sum_{\substack{i\in A\setminus A_{s^*}:\\\Delta_{i}> b}}32K +\sum_{\substack{i\in A \setminus A_{s^*}:\\ 0 < \Delta_{i} \leq b}}32K 
 \!+\! \max\limits_{i:\Delta_{i}\leq b}\Delta_{i}T, 
\end{align*}
where $b\geq \sqrt{\frac{e}{T}}$, and $A_{s^{*}}$ is the subset of arms in cluster $s^{*}$ containing optimal arm $a^{*}$.
%, $\rho_{a}=\frac{1}{2},\rho_{s}=\frac{1}{2}$ and $\psi=K^{2}T$.
\end{theorem}
\begin{proof} The proof of this theorem is given in Appendix \ref{sec:proofTheorem}.
\end{proof}

\textit{Remark:} The most significant term in the bound above is $\sum_{i\in A:\Delta_{i}\geq b}\frac{64\log{\big(T\frac{\Delta_{i}^{2}}{K}\big)}}{\Delta_{i}}$ and hence, the regret upper bound for ClusUCB is of the order $O\bigg(\frac{K\log \big(\frac{T\Delta^{2}}{K}\big)}{\Delta}\bigg)$. Since Corollary \ref{Result:Corollary:1} holds for all $\Delta \geq \sqrt{\frac{e}{T}} $, it can be clearly seen that for all $\sqrt{\frac{e}{T}} \leq \Delta\leq 1$ and $K\geq 2$, the gap-dependent bound is better than that of UCB1, UCB-Improved and MOSS (see Table~\ref{tab:regret-bds}). 


We now show the the gap-independent regret bound of ClusUCB in Corollary \ref{Result:Corollary:1}.


%We now specialize the result in the theorem above by substituting specific values for the exploration constants $\rho_{s}$, $\rho_{a}$ and $\psi$. 


\begin{corollary}[\textbf{\textit{Gap-independent bound}}]
\label{Result:Corollary:1}
Considering the same gap of $\Delta_{i} = \Delta =\sqrt{\frac{K\log K}{T}}$ for all ${i:i\neq *}$ and with $\psi=\frac{T}{K^2}$, $p=\left\lceil\frac{K}{\log K}\right\rceil$, $\rho_{a}=\frac{1}{2}$ and $\rho_{s}=\frac{1}{2}$ and for $T\geq K^{2.4}$, we have the following gap-independent bound for the regret of ClusUCB:
\begin{align*}
\E[R_{T}]\leq 96\sqrt{KT} + 12K^2 + 44K\log K + \dfrac{64 K^3}{K+\log K}
\end{align*}
\end{corollary}

\begin{proof}
The proof of this corollary is given in Appendix~\ref{App:Proof:Corollary:1}
\end{proof}

\textit{Remarks:} From the above result, we observe that the order of the regret upper bound of ClusUCB is $O(\sqrt{KT})$, and this matches the order of MOSS, OUCUCB and UCB-Improved and is order optimal. This  bound is also better than UCB1 and UCB-Improved. 

Next, we state the special case of ClusUCB when $p=1$, i.e there is a single cluster and there are no cluster elimination condition but only arm elimination condition. We name this algorithm ClusUCB-AE.


\begin{proposition}
\label{proofTheorem:Prop:1}
The regret $R_T$ for ClusUCB-AE satisfies
\begin{align*}
&\E [R_{T}]\leq \E[R_{T}] \leq &\sum\limits_{i\in A:\Delta_{i} > b} \left\lbrace 12K + \bigg(\Delta_{i}+\dfrac{32\log{(\frac{T\Delta_i^2}{K})}}{\Delta_{i}}\bigg) + 16K\right\rbrace +\sum\limits_{i\in A:0 < \Delta_{i}\leq b} 16K + \max_{i\in A:\Delta_{i}\leq b}\Delta_{i}T,
\end{align*}
for all $b\geq\sqrt{\frac{e}{T}}$.
\end{proposition}

\begin{proof}
The proof of this proposition is given in Appendix~\ref{App:A}
\end{proof}


%%%%% Gap dependent bound
%\begin{corollary}[\textbf{\textit{Gap-dependent bound}}]
%\label{Result:Corollary:1}
%With $\psi=\frac{T}{196\log (K)}$, $\rho_{a}=\frac{1}{2}$, and $\rho_{s}=\frac{1}{2}$,  we have the following gap-dependent bound for the regret of EClusUCB:
%\begin{align*}
%&\E [R_T] \!\le\! 
%\sum_{\substack{i\in A_{s^{*}}:\\\Delta_{i} > b}}\bigg\lbrace \frac{192\sqrt{\log (K)}}{\Delta_{i}} + \Delta_{i} + 
% \frac{64\log{(T\frac{\Delta_{i}^{2}}{\sqrt{\log (K)}})}}{\Delta_{i}} \bigg\rbrace + \sum_{i\in A:\Delta_{i} > b}\bigg\lbrace\frac{112\sqrt{\log (K)}}{\Delta_{i}} \\
% %%%%%%%%%%%%%%%%%%%%%%
% & + 2\Delta_{i}  + \frac{128\log{(T\frac{\Delta_{i}^{2}}{\sqrt{\log (K)}})}}{\Delta_{i}}\bigg\rbrace
%	  + \sum\limits_{\substack{i\in A_{s^{*}}:\\0< \Delta_{i} \leq b}}\frac{80\sqrt{\log (K)}}{\Delta_{i}}
%	 + \sum\limits_{\substack{i\in A\setminus A_{s^{*}}:\\\Delta_{i} > b}}\frac{160\sqrt{\log (K)}}{\Delta_{i}} \\
%	 %%%%%%%%%%%%%%%%%%%%
%	 & + \sum\limits_{\substack{i\in A\setminus A \cup A_{s^{*}}:\\0 < \Delta_{i}\leq b}}\frac{160\sqrt{\log (K)}}{\Delta_{i}}  + \max\limits_{i\in A:\Delta_{i}\leq b}\Delta_{i}T, \quad \text{ for all }b\geq \sqrt{\frac{K}{14 T}}.
%	\end{align*} 
%\end{corollary}
%\begin{proof}
% See Appendix \ref{App:Proof:Corollary:1}.
%\end{proof}
%
%
%
%
%\begin{corollary}[\textbf{\textit{Gap-independent bound}}]
%\label{Result:Corollary:2}
%Considering the same gap of $\Delta_{i} = \Delta =\sqrt{\frac{K\log K}{T}}$ for all ${i:i\neq *}$ and with $\psi=\frac{T}{196 \log K}$, $p=\left\lceil\frac{K}{\log K}\right\rceil$, $\rho_{a}=\frac{1}{2}$ and $\rho_{s}=\frac{1}{2}$, 
% we have the following gap-independent bound for the regret of EClusUCB:
%\begin{align*}
% \E[R_{T}]\le & 540\frac{\sqrt{T}\log K}{\sqrt{K}} \!+\! \frac{64\sqrt{T\log K}\log{(\log K)}}{\sqrt{K}}\\
%  &\!+\! 112\sqrt{KT} \!+\! 256\sqrt{KT\log K}
%	 + \frac{128\sqrt{KT}\log{(\log K)}}{\sqrt{\log K}} + 300\sqrt{\frac{T\log K}{e}}\\
%%%%%%%%%%%%%%%%%%%%
%	& + 600\sqrt{\frac{T}{e}}(\log K)^{\frac{3}{2}} + 600 \frac{K}{K+\log K}\sqrt{KT}
%\end{align*}
%\end{corollary}
%\begin{proof}
% See Appendix \ref{App:Proof:Corollary:2}.
%\end{proof}


%From the above result, we observe that the order of the regret upper bound of EClusUCB is $O(\sqrt{KT\log K})$, and this matches the order of UCB-Improved. However, this is not as low as the order $O(\sqrt{KT})$ of MOSS or OCUCB. Also, the gap-independent bound of UCB-Improved holds for $ \sqrt{\frac{e}{T}} \leq \Delta \leq 1$ while in our case the gap independent bound holds for $\sqrt{\frac{K}{14T}} \leq \Delta \leq 1$.


\subsection*{Analysis of elimination error (Why Clustering?)}
%\vspace*{-0.4em}
Let $\widetilde R_T$ denote the contribution  to the expected regret in the case when the optimal arm $*$ gets eliminated during one of the rounds of ClusUCB. This can happen if a sub-optimal arm eliminates $*$ or if a sub-optimal cluster eliminates the cluster $s^*$ that contains $*$ -- these correspond to cases b2 and b3 in the proof of Theorem \ref{Result:Theorem:1} (see Section \ref{sec:proofTheorem}). 
As stated before We shall denote variant of ClusUCB that includes arm elimination condition only as ClusUCB-AE while ClusUCB corresponds to Algorithm \ref{alg:clusucb}, which uses both arm and cluster elimination conditions. The regret upper bound for ClusUCB-AE is given in Proposition \ref{proofTheorem:Prop:1}.

For ClusUCB-AE, the quantity $\widetilde R_T$ can be extracted from the proofs (in particular, case b2 in Appendix \ref{App:A}) and simplified to obtain $\widetilde R_T = 32K^2 $. Finally, for ClusUCB, the relevant terms from Theorem \ref{Result:Theorem:1} that corresponds to $\widetilde R_T$ can be simplified with $\rho_{a}=\frac{1}{2}$, $\rho_{s}=\frac{1}{2},p=\big\lceil \frac{K}{\log K} \big\rceil$ and $\psi=\frac{T}{K^2}$ (as in Corollary \ref{Result:Corollary:1} to obtain  
$\tilde R_T = 32K\log K + \dfrac{64 K^3}{K+\log K}$. Hence, in comparison to ClusUCB-AE which has an elimination regret bound of $O(K^2)$, the elimination error regret bound of ClusUCB is lower and of the order $O(\dfrac{64 K^3}{K+\log K})$. Thus, we observe that clustering in conjunction with improved exploration via $\rho_{a},\rho_{s}$,$p$ and $\psi$ helps in reducing the factor associated with $K^2$ for the gap-independent error regret bound for ClusUCB. Also in section \ref{sec:expts}, in experiment $4$ we show that ClusUCB outperforms ClusUCB-AE. 

%A table containing the regret error bound is shown in Appendix \ref{App:E}.


