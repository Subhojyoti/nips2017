\documentclass{article}

% if you need to pass options to natbib, use, e.g.:
% \PassOptionsToPackage{numbers, compress}{natbib}
% before loading nips_2017
%
% to avoid loading the natbib package, add option nonatbib:
% \usepackage[nonatbib]{nips_2017}

\usepackage{nips_2017}

% to compile a camera-ready version, add the [final] option, e.g.:
% \usepackage[final]{nips_2017}

\usepackage[utf8]{inputenc} % allow utf-8 input
\usepackage[T1]{fontenc}    % use 8-bit T1 fonts
\usepackage{hyperref}       % hyperlinks
\usepackage{url}            % simple URL typesetting
\usepackage{booktabs}       % professional-quality tables
\usepackage{amsfonts}       % blackboard math symbols
\usepackage{nicefrac}       % compact symbols for 1/2, etc.
\usepackage{microtype}      % microtypography


\usepackage{macros}

\title{UCB with clustering and improved exploration}

% The \author macro works with any number of authors. There are two
% commands used to separate the names and addresses of multiple
% authors: \And and \AND.
%
% Using \And between authors leaves it to LaTeX to determine where to
% break the lines. Using \AND forces a line break at that point. So,
% if LaTeX puts 3 of 4 authors names on the first line, and the last
% on the second line, try using \AND instead of \And before the third
% author name.

%\author{
%  David S.~Hippocampus\thanks{Use footnote for providing further
%    information about author (webpage, alternative
%    address)---\emph{not} for acknowledging funding agencies.} \\
%  Department of Computer Science\\
%  Cranberry-Lemon University\\
%  Pittsburgh, PA 15213 \\
%  \texttt{hippo@cs.cranberry-lemon.edu} \\
%  %% examples of more authors
%  %% \And
%  %% Coauthor \\
%  %% Affiliation \\
%  %% Address \\
%  %% \texttt{email} \\
%  %% \AND
%  %% Coauthor \\
%  %% Affiliation \\
%  %% Address \\
%  %% \texttt{email} \\
%  %% \And
%  %% Coauthor \\
%  %% Affiliation \\
%  %% Address \\
%  %% \texttt{email} \\
%  %% \And
%  %% Coauthor \\
%  %% Affiliation \\
%  %% Address \\
%  %% \texttt{email} \\
%}

\begin{document}
% \nipsfinalcopy is no longer used

\maketitle

\begin{abstract}
In this paper, we present a novel algorithm for the stochastic multi-armed bandit (MAB) problem. Our proposed Clustered UCB method, referred to as ClusUCB partitions the arms into clusters and then follows the UCB-Improved strategy with aggressive exploration factors to eliminate sub-optimal arms, as well as entire clusters. Through a theoretical analysis, we establish that ClusUCB achieves a better gap-dependent regret upper bound than UCB-Improved~\cite{auer2010ucb} and MOSS~\cite{audibert2009minimax} algorithms. ClusUCB also achieves a gap-independent regret bound of $O\left(\sqrt{KT}\right)$ which is comparable to MOSS and OCUCB~\cite{lattimore2016optimally} and is order optimal. Further, numerical experiments on test-cases with small gaps between optimal and sub-optimal mean rewards show that ClusUCB results in lower cumulative regret than several popular UCB variants as well as MOSS, OCUCB~\cite{lattimore2015optimally}, Thompson sampling and Bayes-UCB\cite{kaufmann2012bayesian}. 
%We also present another algorithm called Adaptive Clustered UCB or AClusUCB which is intended to look at the effect of using more traditional approaches, like hierarchical clustering, for the grouping of arms.

%\keywords{Multi-armed Bandits, Cumulative Regret, Clustering, UCB-Improved}
\end{abstract}

%\vspace*{-3em}
\section{Introduction}
\label{sec:intro}
In this paper, we consider the stochastic multi-armed bandit problem, a classical problem in sequential decision making. In this setting,  a learning algorithm is provided with a set of decisions (or arms) with reward distributions unknown to the algorithm. The learning proceeds in an iterative fashion, where in each round, the algorithm chooses an arm and receives a stochastic reward that is drawn from a stationary distribution specific to the arm selected.  
Given the goal of maximizing the cumulative reward, the learning algorithm faces the exploration-exploitation dilemma, i.e., in each round should the algorithm select the arm which has the highest observed mean reward so far 
(\textit{exploitation}), or should the algorithm choose a new arm to gain more knowledge of the true mean reward of the arms and thereby avert a sub-optimal greedy decision (\textit{exploration}). 

Let $r_i$, $i=1,\ldots,K$ denote the mean reward of the $i$th arm out of the $K$ arms and $r^* = \max_i r_i$ the optimal mean reward. The objective in the stochastic bandit problem is to minimize the cumulative regret, which is defined as follows:
\begin{align*}
R_{T}=r^{*}T - \sum_{i\in A} r_{i}N_{i}(T),
\end{align*}
where $T$ is the number of timesteps, $N_{i}(T)=\sum_{m=1}^T I(I_m=i)$ is the number of times the algorithm has chosen arm $i$ up to timestep $T$.
The expected regret of an algorithm after $T$ timesteps can be written as
%\newline
%\newline
\begin{align*}
\E[R_{T}]= \sum_{i=1}^K \E[N_i(T)] \Delta_i,
\end{align*}
where $\Delta_{i}=r^{*}-r_{i}$ denotes the gap between the means of the optimal arm and the $i$-th arm. 


%The problem, gets more difficult when the $\Delta_{i}$'s are smaller and arm set is larger. Also let $\Delta=min_{i\in A}\Delta_{i}$, that is it is the minimum possible gap over all arms in $A$.
                                                                                                                                          

%NEW RELATED WORK AND CONTRIBUTION

An early work involving a bandit setup is \cite{thompson1933likelihood}, where the author deals with  the problem of choosing between two treatments to administer on patients who come in sequentially. Following the seminal work of \cite{robbins1952some}, bandit algorithms have been extensively studied in a variety of applications. 
From a theoretical standpoint, an asymptotic lower bound for the regret was established in \cite{lai1985asymptotically}. In particular, it was shown that for any consistent allocation strategy, we have
$\liminf_{T \to \infty}\frac{\E[R_{T}]}{\log T}\geq\sum_{\{i:r_{i}<r^{*}\}}\frac{(r^{*}-r_{i})}{D(p_{i}||p^{*})},$
where $D(p_{i}||p^{*})$ is the Kullback-Leibler divergence between the reward densities $p_{i}$ and $p^{*}$, corresponding to arms with mean $r_{i}$ and $r^{*}$, respectively.

	There have been several algorithms with strong regret guarantees. For further reference we point the reader to \cite{bubeck2012bandits}. The foremost among them is UCB1 \cite{auer2002finite}, which has a regret upper bound of $O\big(\frac{K\log T}{\Delta}\big)$, where $\Delta = \min_{i:\Delta_i>0} \Delta_i$. This result is asymptotically order-optimal for the class of distributions considered. However, the worst case gap independent regret bound of UCB1  can be as bad as $O \big(\sqrt{TK\log T}\big)$.  In \cite{audibert2009minimax}, the authors propose the MOSS algorithm and establish that the worst case regret of MOSS is $O\big(\sqrt{TK}\big)$ which improves upon UCB1 by a factor of order $\sqrt{\log T}$. However, the gap-dependent regret of MOSS is  $O\big(\frac{K^{2}\log\left(T\Delta^{2}/K\right)}{\Delta}\big)$ and in certain regimes, this can be worse than even UCB1 (see \cite{audibert2009minimax,lattimore2015optimally}). The UCB-Improved algorithm, proposed in \cite{auer2010ucb}, is a round-based algorithm\footnote{An algorithm is \textit{round-based} if it pulls all the arms equal number of times in each round and then proceeds to eliminate one or more arms that it identifies to be sub-optimal.} variant of UCB1 that 
has a gap-dependent regret bound of $O\big(\frac{K\log T\Delta^{2}}{\Delta}\big)$, which is better than that of UCB1. On the other hand, the worst case regret of UCB-Improved is $O\big(\sqrt{TK\log K}\big)$. Recently in \cite{lattimore2015optimally}, the algorithm OCUCB achieves order-optimal gap-dependent regret bound of $O\left(\sum_{i=2}^{K}\frac{\log\left(T/H_i\right)}{\Delta_i}\right)$ where $H_i=\sum_{j=1}^{K}\min\lbrace \frac{1}{\Delta_i^2},\frac{1}{\Delta_j^2}\rbrace$ and gap-independent regret bound of $O\big( \sqrt{KT}\big)$. 
%In certain environments, for a larger number of arms and uniform gaps, OCUCB does not perform well. 
In certain environments we demonstrate that OCUCB performs poorly. This is specifically true in settings where the gaps between optimal and sub-optimal arms are uniform, which is in line with the observations of \cite{lattimore2015optimally}.

The idea of clustering in the bandit framework is not entirely new. In particular, the idea of clustering has been extensively studied in the contextual bandit setup, an extension of the MAB where side information or features are attached to each arm (see  \cite{auer2002using,langford2008epoch,li2010contextual,beygelzimer2011contextual,slivkins2014contextual}) . The clustering in this case is typically done over the feature space \cite{bui2012clustered,cesa2013gang,gentile2014online}, however, in our work we cluster or group the arms.  

\vspace*{-0.7em}
\subsection{Our Contribution}
We propose a variant of UCB algorithm, called Clustered UCB, henceforth referred to as ClusUCB, that incorporates clustering and an improved exploration scheme. ClusUCB starts with partitioning of arms into small clusters, each having same number of arms. The clustering is done at the start with a prespecified number of clusters. At the end of every round ClusUCB conducts both (individual) arm elimination as well as cluster elimination. This is the first algorithm in bandit literature which uses two simultaneous arm elimination conditions and shows both theoretically and empirically that such an approach is indeed helpful.


%While the conditions governing the arm and cluster eliminations are inspired by UCB-Improved, the exploration factors governing these conditions are relatively more aggressive than that in UCB-Improved. 

The clustering of arms provides two benefits. First, it creates a context where a UCB-Improved like algorithm can be run in parallel on smaller sets of arms with limited exploration, which could lead to fewer pulls of sub-optimal arms with the help of  more aggressive elimination of sub-optimal arms. Second, the cluster elimination leads to whole sets of sub-optimal arms being simultaneously eliminated when they are found to yield poor results. These two simultaneous criteria for arm elimination can be seen as borrowing the strengths of UCB-Improved as well as other popular round based approaches. 

We will also show that in certain environments ClusUCB is able to take advantage of the underlying structure of the reward distribution of arms that other algorithms fail to take advantage of. We will briefly discuss two of these examples here.
%Note that cluster elimination serves as a mode of indirect arm elimination whereby the other arms in the cluster is eliminated based on the performance of the best performing arm in the cluster.
\\
\textit{1.Bernoulli Distribution with small gaps:} In this environment there are $20$ arms with means $r_{1:12}=0.01$, $r_{13:19}=0.07$ and $r_{20}^{*} =0.1$. Here, EClusUCB because of random partitioning of arms into clusters, will create clusters where there are atleast one arm with means $0.07$ and a significant number of arms with $0.01$ means. These clusters behave like independent UCB-Improved algorithms with improved exploration factors and the arms with means $0.01$ are quickly eliminated. Note that since gaps are very small and the gaps of arms with means $0.07$ are very close to the optimal arm, comparing all arms to the single best performing arm at every timestep will result is fewer arm eliminations. Hence utilizing the clusters as in ClusUCB results in faster elimination of arms. This is shown in Experiment 1.\\
\textit{2.Gaussian Distribution with different variances:} In this environment there are $100$ arms with means $r_{1:66}=0.1,\sigma_{1:66}^2 =0.7$, $r_{67:99}=0.8,\sigma_{67:99}^2 =0.1$ and $r_{100}^{*}=0.9, \sigma_{100}^2 =0.7$. Here, the variance of the optimal arm and arms with mean farthest from  the optimal arm are the highest. Whereas, the arms having mean closest to the optimal arm have lowest variances. In these type of cases, due to clustering ClusUCB is able to eliminate the arms with means $0.7$ quickly because clusters containing atleast one arm with $0.8$ mean behaves as independent UCB-Improved algorithms with improved exploration factors. This is shown in Experiment 2. Again, note that due to high variance of the optimal arm, comparing only with the best performing arm at every timestep results in fewer arm eliminations.

Theoretically, while ClusUCB does not achieve the gap-dependent regret bound of OCUCB, the theoretical analysis establishes that the gap-dependent regret of ClusUCB is always better than that of UCB-Improved and better than that of MOSS (see Table~\ref{tab:comp-bds}. Moreover, the gap-independent bound of ClusUCB is of the same order as MOSS and OCUCB, i.e., $O\left(\sqrt{KT}\right)$. 


\begin{table}[t]
\caption{Regret upper bound of different algorithms}
\label{tab:comp-bds}
\begin{center}
\begin{tabular}{p{6em}p{12em}p{10em}}
\toprule
Algorithm  & Gap-Dependent & Gap-Independent \\
\hline
ClusUCB		& $O\left( \dfrac{K\log (T\Delta^2 /K)}{\Delta}\right)$ & $O\left(\sqrt{KT}\right)$\\
UCB1        & $O\left( \dfrac{K\log T}{\Delta} \right)$ & $O\left(\sqrt{KT\log T}\right)$ \\%\midrule
UCB-Imp 		& $O\left( \dfrac{K\log (T\Delta^2)}{\Delta} \right)$ & $O\left(\sqrt{KT\log K}\right)$ \\%\midrule
MOSS	     	& $O\left( \dfrac{K^2\log (T\Delta^2 /K)}{\Delta}\right)$ & $O\left(\sqrt{KT}\right)$\\%\midrule
OCUCB     	& $O\left( \dfrac{K\log (T/ H_{i})}{\Delta}\right)$ & $O\left(\sqrt{KT}\right)$\\\midrule
\end{tabular}
\end{center}
\vspace*{-2em}
\end{table}



%However, EClusUCB is not able to match the gap-independent bound of $O(\sqrt{KT})$ for MOSS and OCUCB. We also establish the exact values for the exploration parameters and the number of clusters required for optimal behavior in the corollaries.
% 	To counter early exploration in \cite{liu2016modification} as well as in our algorithm we propose an exploration regulatory factor to control exploration. \textit{Our algorithm is also not an anytime algorithm}(neither is MOSS, UCB-Improved) and in this context we point out that knowledge of the horizon actually facilitates learning as it can exploit more information(as stated in \cite{lattimore2015optimally}). We also employ a couple of more strategies to bring down our regret as summarized below:-
%\begin{table}
%\caption{Comparison of different algorithms against EClusUCB. The \checkmark indicates that EClusUCB outperforms the respective baseline. E1, E2 and E3 correspond to experiments 1,2 and 3 in Section \ref{sec:expts}}.
%\label{tab:comp-bds}
%\begin{center}
%\begin{tabular}{cccccc}
%\toprule
%Algorithm  & Gap-Dep & Gap-Ind & E1 & E2 & E3\\
%\hline
%UCB1        &\checkmark &\checkmark &\checkmark &\checkmark & N/A \\%\midrule
%UCB-Imp 		&\checkmark &\checkmark &\checkmark &\checkmark & N/A\\%\midrule
%MOSS	     	&\checkmark &\xmark &\checkmark &\checkmark &\checkmark\\%\midrule
%OCUCB     	&\xmark &\xmark &\checkmark &\checkmark &\checkmark\\\midrule
%%EClusUCB      &\checkmark &\checkmark &\checkmark &\checkmark \\\bottomrule
%\end{tabular}
%\end{center}
%\vspace*{-2em}
%\end{table}
%\begin{table}
%\caption{Gap-dependent regret bounds for different bandit algorithms}
%\label{tab:regret-bds}
%\begin{center}
%\begin{tabular}{|c|c|}
%\toprule
%Algorithm  & Upper bound \\
%\midrule
%UCB1         &$O\left(\frac{K\log T}{\Delta}\right)$ \\\midrule
%UCB-Improved &$O\left(\frac{K\log (T\Delta^{2})}{\Delta}\right)$ \\\midrule
%MOSS	     &$O\left(\frac{K^{2}\log\left(T\Delta^{2}/K\right)}{\Delta}\right)$\\\midrule
%ClusUCB$\big /$EClusUCB      &$O\left(\frac{K\log\left(\frac{T\Delta^{2}}{\sqrt{\log (K)}}\right)}{\Delta}\right)$\\\bottomrule
%\end{tabular}
%\end{center}
%\vspace*{-2em}
%\end{table}
%While ClusUCB is a round-based algorithm, we also introduce Efficient ClusUCB or EClusUCB which has the same theoretical guarantees as ClusUCB but empirically behaves much better (see Table~\ref{tab:comp-bds}). 
On four synthetic setups with small gaps, we observe empirically that EClusUCB outperforms UCB-Improved\cite{auer2010ucb}, MOSS\cite{audibert2009minimax} and OCUCB\cite{lattimore2015optimally} as well as other popular stochastic bandit algorithms such as UCB-V\cite{audibert2009exploration}, Median Elimination\cite{even2006action}, Thompson Sampling\cite{agrawal2011analysis}, Bayes-UCB\cite{kaufmann2012bayesian} and KL-UCB\cite{garivier2011kl}. 
%Adaptive ClusUCB (AClusUCB) which estimates the clusters based on hierarchical clustering is introduced in Appendix \ref{App:AClusUCB}. An empirical study comparing its performance to EClusUCB is presented in experiment $4$. 
%DMED\cite{honda2010asymptotically},

The rest of the paper is organized as follows: In Section \ref{sec:clusucb} we introduce ClusUCB. In Section \ref{sec:results}, we present the associated regret bounds. In Section \ref{sec:expts}, we present the numerical experiments and provide concluding remarks in Section \ref{sec:conclusions}. Further proofs of lemmas, corollaries, theorems and propositions presented in Section \ref{sec:results}  are provided in the appendices. 

%More experiments are presented in Appendix \ref{App:MoreExp}. 
%and prove the main theorem on the regret upper bound for ClusUCB in Section \ref{sec:proofTheorem}
%and we also show in Appendix \ref{App:MoreExp} that EClusUCB which employs an uniform clustering scheme performs better then AClusUCB
%Appendix \ref{App:A}, \ref{App:B} deals with proofs of  2  propositions which are derived from our main regret bound theorem. Appendix \ref{App:Proof:Corollary:1}, \ref{App:Proof:Corollary:2}, \ref{App:Proof:Corollary:3} deals with proofs of 3 corollaries which specializes the result of our main regret bound theorem. Appendix \ref{App:D} deals with exploration regulatory factor and appendix \ref{App:E} explains why we do clustering. The last appendix \ref{App:Further:Expt} deals with further experiments on two different testbeds.


\section{Algorithm: Clustered UCB}
\label{sec:clusucb}
\paragraph{Notation.}
\label{sec:prelims}
We denote the set of arms by $A$, with the individual arms labeled $i, i=1,\ldots,K$.
We denote an arbitrary round of ClusUCB by $m$. We denote an arbitrary cluster by $s_{k}$, the subset of arms within the cluster $s_k$ by  $A_{s_{k}}$ and the set of clusters by $S$ with $|S|=p\leq K$. 
Here $p$ is a pre-specified limit for the number of clusters.
For simplicity, we assume that the optimal arm is unique and denote it by ${*}$, with $s^{*}$ denoting the corresponding cluster.
 %and the subset of arms in $s^{*}$ is denoted by $A_{s^{*}}$. 
The best arm in a cluster $s_{k}$ is denoted by $a_{max_{s_{k}}}$.  
We denote the sample mean of the rewards seen so far for arm $i$ by $\hat{r_i}$ and for the true best arm within a cluster $s_k$ by $\hat{r}_{a_{\max_{s_{k}}}}$. $z_i$ is the number of times an arm $i$ has been pulled.
%The exploration regulatory factor is denoted by $\psi$. The arm and cluster elimination parameters are denoted by $\rho_{a}$ and $\rho_{s}$ respectively. 
%$\Delta_{i}^{'}=r_{a_{\max_{s_{k}}}} - r_{i}$, such that $a_{i}\in s_{k}$.
% The variable $B_{m}$ denotes the arm set containing the arms that are not eliminated till round $m$.
%  The exploration regulatory factor is denoted by $\psi_{m}$. %\todos{\textit{``$\hat{r}_{min_{s_{i}}}\in s_{i}$ as the arm with minimum estimated payoff''}. $\hat{r}_{min_{s_{i}}}\in s_{i}$ is the minimum estimated payoff and not the arm corresponding to min est payoff. Similar writing pops up at several other places, for e.g., in the statements of the propositions. (Subho) Changed the line as saying the minimum/maximum estimated payoff}
%The parameter $\psi(m)$ is a monotonically decreasing function over the rounds, that is $\psi(m+1)\leq \psi(m)$. 
%Also, we define $w\geq 2$, a weight factor.
We assume that the rewards of all arms are bounded in $[0,1]$.


%worst arm within a cluster $s_i$ by $\hat{r}_{min_{s_{k}}}\in s_{k}$

\subsubsection*{The algorithm}


%%%%%%%%%%%%%%%% alg-custom-block %%%%%%%%%%%%
\algblock{ArmElim}{EndArmElim}
\algnewcommand\algorithmicArmElim{\textbf{\em Arm Elimination}}
 \algnewcommand\algorithmicendArmElim{}
\algrenewtext{ArmElim}[1]{\algorithmicArmElim\ #1}
\algrenewtext{EndArmElim}{\algorithmicendArmElim}

\algblock{ClusElim}{EndClusElim}
\algnewcommand\algorithmicClusElim{\textbf{\em Cluster Elimination}}
 \algnewcommand\algorithmicendClusElim{}
\algrenewtext{ClusElim}[1]{\algorithmicClusElim\ #1}
\algrenewtext{EndClusElim}{\algorithmicendClusElim}
\algtext*{EndArmElim}
\algtext*{EndClusElim}

\algblock{ResParam}{EndResParam}
\algnewcommand\algorithmicResParam{\textbf{\em Reset Parameters}}
 \algnewcommand\algorithmicendResParam{}
\algrenewtext{ResParam}[1]{\algorithmicResParam\ #1}
\algrenewtext{EndResParam}{\algorithmicendResParam}

\begin{algorithm}[t]
\caption{ClusUCB}
\label{alg:clusucb}
\begin{algorithmic}
\State {\bf Input:} Number of clusters $p$, time horizon $T$, exploration parameters $\rho_a$, $\rho_s$ and $\psi$.
\State {\bf Initialization:} Set $B_{0}:=A$, $S_0 = S$ and $\epsilon_{0}:=1$.
\State Create a partition $S_0$ of the arms at random into $p$ clusters of size up to $\ell=\bigg\lceil \dfrac{K}{p} \bigg\rceil$ each.
\For{$m=0,1,..\big \lfloor \dfrac{1}{2}\log_{2} \dfrac{T}{e}\big\rfloor$}	
\State Pull each arm in $B_m$ so that the total number of times it has been pulled is $n_{m}=\bigg\lceil\dfrac{\log{(\psi T\epsilon_{m}^{2})}}{2\epsilon_{m}}\bigg\rceil$. 
% A partition of $A$ into clusters from Algorithm \ref{alg:rua}
%\State \hspace*{2em} Calculate $w_{s_{i}}=\bigg\lceil\dfrac{1}{\ell\hat{\Delta}_{s_{i}}}\bigg\rceil$,if $\hat{\Delta}_{s_{i}}\neq 0, \forall s_{i}\in S$
%\newline\hspace*{8em}$=1$, otherwise, and $\hat{\Delta}_{s_{i}}=\max_{i\in s_{i}}{\lbrace\hat{r}_{i}\rbrace}-\min_{j\in s_{i}}{\lbrace\hat{r}_{j}\rbrace}, i\neq j$
\ArmElim
\State For each cluster $s_k \in S_{m}$, delete arm ${i}\in s_{k}$ from $B_{m}$ if
\begin{align*}
\hat{r}_{i} + \sqrt{\dfrac{\rho_{a}\log{(\psi T\epsilon_{m})}}{2 n_{m}}}  < \max_{{j}\in s_{k}}\bigg\lbrace\hat{r}_{j} -\sqrt{\dfrac{\rho_{a}\log{(\psi T\epsilon_{m})}}{2 n_{m}}} \bigg\rbrace
\end{align*}
% where $\rho_{a}=\dfrac{1}{w_{m}}$ and remove all such arms from $B_{m}$.
\EndArmElim
\ClusElim
\State Delete cluster $s_{k}\in S_{m}$ and remove all arms $i\in s_{k}$ from $B_{m}$ if 
\begin{align*}
 \max_{{i}\in s_{k}}\bigg\lbrace\hat{r}_{i} + \sqrt{\dfrac{\rho_{s}\log{(\psi T\epsilon_{m})}}{2 n_{m}}}\bigg\rbrace  
 < \max_{{j}\in B_{m}} \bigg\lbrace\hat{r}_{j} - \sqrt{\dfrac{\rho_{s} \log{(\psi T\epsilon_{m})}}{2 n_{m}}}\bigg\rbrace.
\end{align*}
%  and remove all such arms in the cluster $s_{k}$ from $B_{m}$ to obtain $B_{m+1}$.
\EndClusElim
\State Set $\epsilon_{m+1}:=\dfrac{\epsilon_{m}}{2}$\vspace{0.5ex}
\State Set $B_{m+1}:=B_{m}$
\State Stop if $|B_{m}|=1$ and pull ${i}\in B_{m}$ till $T$ is reached.
\EndFor
\end{algorithmic}
\end{algorithm}

%\todos[inline]{Shouldn't there be a $\psi$ inside the log term on RHS of both elim conditions of Algorithm \ref{alg:clusucb}? (Subho) Addressed: $\psi$ has to be there}

As mentioned in a recent work \cite{liu2016modification}, UCB-Improved has two shortcomings: 	\\
\begin{inparaenum}[\bfseries(i)]
\item A significant number of pulls are spent in early exploration, since each round $m$ of UCB-Improved involves pulling every arm an identical $n_{m}=\bigg\lceil \dfrac{ 2\log(T\epsilon^{2}_{m})}{\epsilon^{2}_{m}} \bigg\rceil$ number of times. The quantity $\epsilon_{m}$ is initialized to $1$ and halved after every round.\\
\item In UCB-Improved, arms are eliminated conservatively, i.e, only after $\epsilon_{m}<\dfrac{\Delta_{i}}{2}$, the sub-optimal arm $i$ is discarded with high probability. This is disadvantageous when $K$ is large and the gaps are identical ($r_{1}=r_{2}=..=r_{K-1}<r^{*}$) and small.\\
\end{inparaenum}
To reduce early exploration, the number $n_m$ of times each arm is pulled per round in ClusUCB is lower than that of UCB-Improved and also that of Median-Elimination, which used $n_m=\dfrac{4}{\epsilon^{2}}\log\big(\dfrac{3}{\delta}\big)$, where $\epsilon,\delta$ are confidence parameters.
To handle the second problem mentioned above, ClusUCB partitions the larger problem into several small sub-problems using clustering and then performs local exploration aggressively to eliminate sub-optimal arms within each clusters with high probability.


As described in the pseudocode in Algorithm~\ref{alg:clusucb}, ClusUCB begins with a initial clustering of arms that is performed by random uniform allocation. The set of clusters $S$ thus obtained satisfies $|S|=p$, with individual clusters having a size that is bounded above by $\ell=\bigg\lceil \dfrac{K}{p} \bigg\rceil$.
Each round of ClusUCB involves both individual arm as well as cluster elimination conditions. These elimination conditions are inspired by UCB-Improved. Notice that, unlike UCB-Improved, there is no longer a single point of reference based on which we are eliminating arms. Instead now we have as many reference points to eliminate arms as number of clusters formed. 
%Further, the exploration factors $\rho_{a}\in (0,1]$ and $\rho_{s}\in (0,1]$ governing the arm and cluster elimination conditions, respectively, are relatively more aggressive than that in UCB-Improved. 

The exploration regulatory factor $\psi$ governing the arm and cluster elimination conditions in ClusUCB is more aggressive than that in UCB-Improved. With appropriate choice of $\psi$ and $\rho_a$ and $\rho_s$ we can achieve aggressive elimination even when the gaps $\Delta_i$ are small and $K$ is large. 

%and the gaps $\Delta_i$ are small, it is efficient to remove sub-optimal arms quickly. 

In \cite{liu2016modification}, the authors recommend incorporating a factor of $d_i$ inside the log-term of the UCB values, i.e., $\max \lbrace\hat{r}_{i}+\sqrt{\frac{d_{i}\log T{\epsilon}_{m}^{2}}{2n_{m}}}\rbrace$. 
The authors there examine the following choices for $d_i$: $\frac{T}{t_{i}}$, $\frac{\sqrt{T}}{t_{i}}$ and $\frac{\log T}{t_{i}}$, where $t_{i}$ is the number of times an arm ${i}$ has been sampled.
Unlike \cite{liu2016modification}, we employ cluster as well as arm elimination and establish from a theoretical analysis that the choice $\psi=\frac{T}{\log (KT)}$ helps in achieving a better gap-dependent regret upper bound for ClusUCB as compared to UCB-Improved and MOSS (see Corollary \ref{Result:Corollary:1} in the next section). 






\section{Main results}
\label{sec:results}
	
We now state the main result that upper bounds the expected regret of ClusUCB.
\begin{theorem}[\textbf{\textit{Gap dependent regret bound}}]
\label{Result:Theorem:1}
For $T\geq K^{2.4} $, $\rho_a =\frac{1}{2}$, $\rho_s =\frac{1}{2}$ and $\psi=\frac{T}{K^2}$ the regret $R_T$ of ClusUCB satisfies
\begin{align*}
&\E [R_{T}]\leq 
\sum\limits_{\substack{i\in A_{s^{*}},\\\Delta_{i} > b}}\bigg\lbrace \Delta_{i} + 12K
+ \frac{32\log{(\frac{T\Delta_i^2}{K})}}{\Delta_{i}} \bigg\rbrace
 + \! \! \sum\limits_{\substack{i\in A,\\\Delta_{i} > b}} \bigg\lbrace 2\Delta_{i} +
12K + \frac{64\log{(\frac{T\Delta_i^2}{K})}}{\Delta_{i}} \bigg\rbrace \\
%%%%%%%%%%%%%%%%%
&+ \sum\limits_{\substack{i\in A_{s^{*}},\\ \Delta_{i} > b}} 
16K+\sum\limits_{\substack{i\in A_{s^{*}},\\0 < \Delta_{i}\leq b}} 16K + \sum_{\substack{i\in A\setminus A_{s^*}:\\\Delta_{i}> b}}32K +\sum_{\substack{i\in A \setminus A_{s^*}:\\ 0 < \Delta_{i} \leq b}}32K 
 \!+\! \max\limits_{i:\Delta_{i}\leq b}\Delta_{i}T, 
\end{align*}
where $b\geq \sqrt{\frac{e}{T}}$, and $A_{s^{*}}$ is the subset of arms in cluster $s^{*}$ containing optimal arm $a^{*}$.
%, $\rho_{a}=\frac{1}{2},\rho_{s}=\frac{1}{2}$ and $\psi=K^{2}T$.
\end{theorem}
\begin{proof} The proof of this theorem is given in Appendix \ref{sec:proofTheorem}.
\end{proof}

\textit{Remark:} The most significant term in the bound above is $\sum_{i\in A:\Delta_{i}\geq b}\frac{64\log{\big(T\frac{\Delta_{i}^{2}}{K}\big)}}{\Delta_{i}}$ and hence, the regret upper bound for ClusUCB is of the order $O\bigg(\frac{K\log \big(\frac{T\Delta^{2}}{K}\big)}{\Delta}\bigg)$. Since Corollary \ref{Result:Corollary:1} holds for all $\Delta \geq \sqrt{\frac{e}{T}} $, it can be clearly seen that for all $\sqrt{\frac{e}{T}} \leq \Delta\leq 1$ and $K\geq 2$, the gap-dependent bound is better than that of UCB1, UCB-Improved and MOSS (see Table~\ref{tab:regret-bds}). 


We now show the the gap-independent regret bound of ClusUCB in Corollary \ref{Result:Corollary:1}.


%We now specialize the result in the theorem above by substituting specific values for the exploration constants $\rho_{s}$, $\rho_{a}$ and $\psi$. 


\begin{corollary}[\textbf{\textit{Gap-independent bound}}]
\label{Result:Corollary:1}
Considering the same gap of $\Delta_{i} = \Delta =\sqrt{\frac{K\log K}{T}}$ for all ${i:i\neq *}$ and with $\psi=\frac{T}{K^2}$, $p=\left\lceil\frac{K}{\log K}\right\rceil$, $\rho_{a}=\frac{1}{2}$ and $\rho_{s}=\frac{1}{2}$ and for $T\geq K^{2.4}$, we have the following gap-independent bound for the regret of ClusUCB:
\begin{align*}
\E[R_{T}]\leq 96\sqrt{KT} + 12K^2 + 44K\log K + \dfrac{64 K^3}{K+\log K}
\end{align*}
\end{corollary}

\begin{proof}
The proof of this corollary is given in Appendix~\ref{App:Proof:Corollary:1}
\end{proof}

\textit{Remarks:} From the above result, we observe that the order of the regret upper bound of ClusUCB is $O(\sqrt{KT})$, and this matches the order of MOSS, OUCUCB and UCB-Improved and is order optimal. This  bound is also better than UCB1 and UCB-Improved. 

Next, we state the special case of ClusUCB when $p=1$, i.e there is a single cluster and there are no cluster elimination condition but only arm elimination condition. We name this algorithm ClusUCB-AE.


\begin{proposition}
\label{proofTheorem:Prop:1}
The regret $R_T$ for ClusUCB-AE satisfies
\begin{align*}
&\E [R_{T}]\leq \E[R_{T}] \leq &\sum\limits_{i\in A:\Delta_{i} > b} \left\lbrace 12K + \bigg(\Delta_{i}+\dfrac{32\log{(\frac{T\Delta_i^2}{K})}}{\Delta_{i}}\bigg) + 16K\right\rbrace +\sum\limits_{i\in A:0 < \Delta_{i}\leq b} 16K + \max_{i\in A:\Delta_{i}\leq b}\Delta_{i}T,
\end{align*}
for all $b\geq\sqrt{\frac{e}{T}}$.
\end{proposition}

\begin{proof}
The proof of this proposition is given in Appendix~\ref{App:A}
\end{proof}


%%%%% Gap dependent bound
%\begin{corollary}[\textbf{\textit{Gap-dependent bound}}]
%\label{Result:Corollary:1}
%With $\psi=\frac{T}{196\log (K)}$, $\rho_{a}=\frac{1}{2}$, and $\rho_{s}=\frac{1}{2}$,  we have the following gap-dependent bound for the regret of EClusUCB:
%\begin{align*}
%&\E [R_T] \!\le\! 
%\sum_{\substack{i\in A_{s^{*}}:\\\Delta_{i} > b}}\bigg\lbrace \frac{192\sqrt{\log (K)}}{\Delta_{i}} + \Delta_{i} + 
% \frac{64\log{(T\frac{\Delta_{i}^{2}}{\sqrt{\log (K)}})}}{\Delta_{i}} \bigg\rbrace + \sum_{i\in A:\Delta_{i} > b}\bigg\lbrace\frac{112\sqrt{\log (K)}}{\Delta_{i}} \\
% %%%%%%%%%%%%%%%%%%%%%%
% & + 2\Delta_{i}  + \frac{128\log{(T\frac{\Delta_{i}^{2}}{\sqrt{\log (K)}})}}{\Delta_{i}}\bigg\rbrace
%	  + \sum\limits_{\substack{i\in A_{s^{*}}:\\0< \Delta_{i} \leq b}}\frac{80\sqrt{\log (K)}}{\Delta_{i}}
%	 + \sum\limits_{\substack{i\in A\setminus A_{s^{*}}:\\\Delta_{i} > b}}\frac{160\sqrt{\log (K)}}{\Delta_{i}} \\
%	 %%%%%%%%%%%%%%%%%%%%
%	 & + \sum\limits_{\substack{i\in A\setminus A \cup A_{s^{*}}:\\0 < \Delta_{i}\leq b}}\frac{160\sqrt{\log (K)}}{\Delta_{i}}  + \max\limits_{i\in A:\Delta_{i}\leq b}\Delta_{i}T, \quad \text{ for all }b\geq \sqrt{\frac{K}{14 T}}.
%	\end{align*} 
%\end{corollary}
%\begin{proof}
% See Appendix \ref{App:Proof:Corollary:1}.
%\end{proof}
%
%
%
%
%\begin{corollary}[\textbf{\textit{Gap-independent bound}}]
%\label{Result:Corollary:2}
%Considering the same gap of $\Delta_{i} = \Delta =\sqrt{\frac{K\log K}{T}}$ for all ${i:i\neq *}$ and with $\psi=\frac{T}{196 \log K}$, $p=\left\lceil\frac{K}{\log K}\right\rceil$, $\rho_{a}=\frac{1}{2}$ and $\rho_{s}=\frac{1}{2}$, 
% we have the following gap-independent bound for the regret of EClusUCB:
%\begin{align*}
% \E[R_{T}]\le & 540\frac{\sqrt{T}\log K}{\sqrt{K}} \!+\! \frac{64\sqrt{T\log K}\log{(\log K)}}{\sqrt{K}}\\
%  &\!+\! 112\sqrt{KT} \!+\! 256\sqrt{KT\log K}
%	 + \frac{128\sqrt{KT}\log{(\log K)}}{\sqrt{\log K}} + 300\sqrt{\frac{T\log K}{e}}\\
%%%%%%%%%%%%%%%%%%%%
%	& + 600\sqrt{\frac{T}{e}}(\log K)^{\frac{3}{2}} + 600 \frac{K}{K+\log K}\sqrt{KT}
%\end{align*}
%\end{corollary}
%\begin{proof}
% See Appendix \ref{App:Proof:Corollary:2}.
%\end{proof}


%From the above result, we observe that the order of the regret upper bound of EClusUCB is $O(\sqrt{KT\log K})$, and this matches the order of UCB-Improved. However, this is not as low as the order $O(\sqrt{KT})$ of MOSS or OCUCB. Also, the gap-independent bound of UCB-Improved holds for $ \sqrt{\frac{e}{T}} \leq \Delta \leq 1$ while in our case the gap independent bound holds for $\sqrt{\frac{K}{14T}} \leq \Delta \leq 1$.


\subsection*{Analysis of elimination error (Why Clustering?)}
%\vspace*{-0.4em}
Let $\widetilde R_T$ denote the contribution  to the expected regret in the case when the optimal arm $*$ gets eliminated during one of the rounds of ClusUCB. This can happen if a sub-optimal arm eliminates $*$ or if a sub-optimal cluster eliminates the cluster $s^*$ that contains $*$ -- these correspond to cases b2 and b3 in the proof of Theorem \ref{Result:Theorem:1} (see Section \ref{sec:proofTheorem}). 
As stated before We shall denote variant of ClusUCB that includes arm elimination condition only as ClusUCB-AE while ClusUCB corresponds to Algorithm \ref{alg:clusucb}, which uses both arm and cluster elimination conditions. The regret upper bound for ClusUCB-AE is given in Proposition \ref{proofTheorem:Prop:1}.

For ClusUCB-AE, the quantity $\widetilde R_T$ can be extracted from the proofs (in particular, case b2 in Appendix \ref{App:A}) and simplified to obtain $\widetilde R_T = 32K^2 $. Finally, for ClusUCB, the relevant terms from Theorem \ref{Result:Theorem:1} that corresponds to $\widetilde R_T$ can be simplified with $\rho_{a}=\frac{1}{2}$, $\rho_{s}=\frac{1}{2},p=\big\lceil \frac{K}{\log K} \big\rceil$ and $\psi=\frac{T}{K^2}$ (as in Corollary \ref{Result:Corollary:1} to obtain  
$\tilde R_T = 32K\log K + \dfrac{64 K^3}{K+\log K}$. Hence, in comparison to ClusUCB-AE which has an elimination regret bound of $O(K^2)$, the elimination error regret bound of ClusUCB is lower and of the order $O(\dfrac{64 K^3}{K+\log K})$. Thus, we observe that clustering in conjunction with improved exploration via $\rho_{a},\rho_{s}$,$p$ and $\psi$ helps in reducing the factor associated with $K^2$ for the gap-independent error regret bound for ClusUCB. Also in section \ref{sec:expts}, in experiment $4$ we show that ClusUCB outperforms ClusUCB-AE. 

%A table containing the regret error bound is shown in Appendix \ref{App:E}.





\section{Simulation experiments}
\label{sec:expts}
We conduct an empirical performance using cumulative regret as the metric. We implement the following algorithms:  KL-UCB\cite{garivier2011kl}, MOSS\cite{audibert2009minimax}, UCB1\cite{auer2002finite}, UCB-Improved\cite{auer2010ucb}, Median Elimination\cite{even2006action}, Thompson Sampling(TS)\cite{agrawal2011analysis}, OCUCB\cite{lattimore2015optimally}, Bayes-UCB(BU)\cite{kaufmann2012bayesian} and UCB-V\cite{audibert2009exploration}\footnote{The implementation for KL-UCB, Bayes-UCB and DMED were taken from \cite{CapGarKau12}}. The parameters of EClusUCB algorithm for all the experiments are set as follows: $\psi=\frac{T}{K^2}$, $\rho_{s}=0.5$, $\rho_{a}=0.5$ and $p=\lceil\frac{K}{\log K}\rceil$ (as in Corollary \ref{Result:Corollary:1}).
%whereas for ClusUCB(Aggressive)$\big/$EClusUCB(Aggressive) or ClusUCBA$\big/$EClusUCBA algorithm the parameters are set as $\psi=\log T$, $\rho_{s}=0.5$, $\rho_{a}=0.5$ and $p=\lceil\frac{K}{\log K}\rceil$. We can clearly see that EClusUCBA has a lower exploration regulatory factor and conducts less exploration and is a riskier algorithm than EClusUCB.
%DMED\cite{honda2010asymptotically}

\begin{figure}[!h]
    \centering
    \begin{tabular}{cc}
    \setlength{\tabcolsep}{0.1pt}
    \subfigure[0.25\textwidth][Expt-$1$: $20$ Bernoulli-distributed arms.]
    %with $r_{i_{{i}\neq {*}}}=0.07$ and $r^{*}=0.1$
    {
    		\pgfplotsset{
		tick label style={font=\Huge},
		label style={font=\Huge},
		legend style={font=\Large},
		ylabel style={yshift=32pt},
		%legend style={legendshift=32pt},
		}
        \begin{tikzpicture}[scale=0.4]
      	\begin{axis}[
		xlabel={timestep},
		ylabel={Cumulative Regret},
		grid=major,
        %clip mode=individual,grid,grid style={gray!30},
        clip=true,
        %clip mode=individual,grid,grid style={gray!30},
  		legend style={at={(0.5,1.5)},anchor=north, legend columns=3} ]
      	% UCB
		\addplot table{results/NewExpt1/Expt1/UCBV01_comp_subsampled.txt};
		%\addplot table{results/NewExpt/Expt1/EclUCB01_1_comp_subsampled.txt};
		\addplot table{results/NewExpt1/Expt1/NEclUCB01_comp_subsampled.txt};
		\addplot table{results/NewExpt1/Expt1/KLUCB01_comp_subsampled.txt};
		\addplot table{results/NewExpt1/Expt1/MOSS01_comp_subsampled.txt};
		%\addplot table{results/NewExpt1/Expt1/DMED01_comp_subsampled.txt};
		\addplot table{results/NewExpt1/Expt1/UCB01_comp_subsampled.txt};
		\addplot table{results/NewExpt1/Expt1/TS01_comp_subsampled.txt};
		\addplot table{results/NewExpt1/Expt1/OCUCB01_comp_subsampled.txt};
		\addplot table{results/NewExpt1/Expt1/BU01_comp_subsampled.txt};
      	%\legend{UCB-V,EClusUCB,KL-UCB,MOSS,DMED,UCB1,TS,OCUCB,BU} 
      	\legend{UCB-V,EClusUCB,KL-UCB,MOSS,UCB1,TS,OCUCB,BU}      	
      	\end{axis}
      	\end{tikzpicture}
  		\label{fig:1}
    }
    &
    \subfigure[0.25\textwidth][Expt-$2$: $100$ Gaussian-distributed arms. ]
    %with $r_{i_{{i}\neq {*}:1-33}}=0.1$, $r_{i_{{i}\neq {*}:34-99}}=0.6$, $r^{*}_{i=100}=0.9$ and $\sigma_{i=1:100}^{2} = 0.3$
    {
    		\pgfplotsset{
		tick label style={font=\Huge},
		label style={font=\Huge},
		legend style={font=\Large},
		ylabel style={yshift=32pt},
		}
        \begin{tikzpicture}[scale=0.4]
        \begin{axis}[
		xlabel={timestep},
		ylabel={Cumulative Regret},
        %clip mode=individual,grid,grid style={gray!30},
       	grid=major,
       	clip=true,
  		legend style={at={(0.5,1.5)},anchor=north, legend columns=3} ]
      	% UCB
        \addplot table{results/NewExpt1/Expt2/UCB01_comp_subsampled.txt};
		%\addplot table{results/NewExpt/Expt2_2/clUCB01_comp_subsampled.txt};
		\addplot table{results/NewExpt1/Expt2/NEclUCB01_comp_subsampled.txt};
		\addplot table{results/NewExpt1/Expt2/MOSS01_comp_subsampled.txt};
		\addplot table{results/NewExpt1/Expt2/OCUCB01_comp_subsampled.txt};
		%\addplot table{results/NewExpt1/Expt2/MedElim01_comp_subsampled.txt};
		\addplot table{results/NewExpt1/Expt2/TS01_comp_subsampled.txt};
		%\addplot table{results/NewExpt1/Expt2/UCBR01_comp_subsampled.txt};
		%\addplot table{results/NewExpt/Expt2_2/EclUCB01_p_1_comp_subsampled.txt};
		%\legend{UCB1,ClusUCB,Med-Elim,MOSS,OCUCB,EClusUCB,UCB-Imp}
		\legend{UCB1,EClusUCB,MOSS,OCUCB,TS}
      	\end{axis}
      	\end{tikzpicture}
   		\label{fig:2}
    }
    \end{tabular}
    \caption{Cumulative regret for various bandit algorithms on two stochastic K-armed bandit environments. }
    \label{fig:karmed}
    \vspace*{-1em}
\end{figure}
% For the purpose of performance comparison


\textbf{First experiment (Bernoulli with small gaps) :} This is conducted over a testbed of $20$ arms in an environment involving Bernoulli reward distributions with expected rewards of the arms $r_{i_{{i}\neq {*}}}=0.07$ and $r^{*}=0.1$. These type of cases are frequently encountered in web-advertising domain. The horizon $T$ is set to $60000$. 
%After limited exploratory experimentation the number of clusters $p$ for ClusUCB is set to $4$. 
The regret is averaged over $100$ independent runs and is shown in Figure \ref{fig:1}. EClusUCB, MOSS, UCB1, UCB-V, KL-UCB, TS, BU and DMED are run in this experimental setup and we observe that EClusUCB performs better than all the aforementioned algorithms except TS. Because of the small gaps and short horizon $T$, we do not implement UCB-Improved and Median Elimination on this test-case. 
%We also observe that in this case the cumulative regret of EClusUCB and TS are almost similar to each other.

\textbf{Second experiment (Gaussian with different variances):} This is conducted over a testbed of $100$ arms involving Gaussian reward distributions with expected rewards of the arms $r_{1:33}=0.7$, $r_{34:99}=0.8$ and $r^{*}_{100}=0.9$ with variance set at $\sigma_{1:33}^{2} = 0.7,\sigma_{34:99}^{2} = 0.1$ and $r^{*}_{100}=0.7$. The horizon $T$ is set for a large duration of $3\times 10^{5}$ and the regret is averaged over $100$ independent runs and is shown in Figure \ref{fig:2}. From the results in Figure \ref{fig:2}, we observe that EClusUCB outperforms MOSS, UCB1, UCB-Improved and Median-Elimination($\epsilon=0.1,\delta=0.1$). Also the performance of UCB-Improved is poor in comparison to other algorithms, which is probably because of pulls wasted in initial exploration whereas EClusUCB with the choice of $\psi, \rho_{a}$ and $\rho_{s}$ performs much better. Note that the performance of TS is poor and this is in line with the observation in \cite{lattimore2015optimally} that the worst case regret of TS in Gaussian distributions is $\Omega\left( \sqrt{KT\log T}\right)$.

\begin{figure}[!h]
    \centering
    \begin{tabular}{cc}
    \subfigure[0.32\textwidth][Expt-$3$: $20$ to $100$ Bernoulli-distributed arms. ]
    %with $r_{i_{{i}\neq {*}}}=0.05$ and $r^{*}=0.1$
    {
    	\pgfplotsset{
		tick label style={font=\Huge},
		label style={font=\Huge},
		legend style={font=\Large},
		ylabel style={yshift=32pt},
		}
        \begin{tikzpicture}[scale=0.4]
        \begin{axis}[
		xlabel={Arms},
		ylabel={Cumulative Regret},
        %clip mode=individual,grid,grid style={gray!30},
		grid=major,
		clip=true,
  		legend style={at={(0.5,1.3)},anchor=north, legend columns=3} ]
        % UCB
		\addplot table{results/NewExpt/Expt3_1/plotFinalMOSS20_100.txt};
		\addplot table{results/NewExpt/Expt3_1/plotFinalFEclUCB20_100.txt};
		\addplot table{results/NewExpt/Expt3_1/plotFinalOCUCB20_100.txt};
      	\legend{MOSS,EClusUCB,OCUCB}
      	\end{axis}
        \end{tikzpicture}
        \label{fig:3}
    }
    &
    \subfigure[0.32\textwidth][Expt-$4$: Different cluster experiment]
    {
    		\pgfplotsset{
		tick label style={font=\Huge},
		label style={font=\Huge},
		legend style={font=\Large},
		ylabel style={yshift=32pt},
		}
        \begin{tikzpicture}[scale=0.4]
      	\begin{axis}[
		ylabel={Cumulative Regret},
		xlabel={Clusters},
		grid=major,
        %clip mode=individual,grid,grid style={gray!30},
        clip=true,
        %clip mode=individual,grid,grid style={gray!30},
  		legend style={at={(0.5,1.3)},anchor=north, legend columns=3} ]
      	% UCB
		%\addplot table{results/NewExpt/Expt4/plotFinalAclUCB012.txt};
		\addplot table{results/NewExpt/Expt4/plotFinalFEclUCB02.txt};
		%\addplot table{results/NewExpt/Expt4/plotFinalEclUCB_AE012.txt};
      	%\legend{EClusUCBA,AClusUCBA,EClusUCB,AClusUCB} 
      	\legend{EClusUCB} 
      	\end{axis}
      	\draw [red,thick] (.61,5.25) circle (0.3 cm);
      	code={
        \node[black,above] at (axis cs:3.0,4.27){\tiny{EClusUCB-AE}};
    		}
      	\end{tikzpicture}
  		\label{fig:4}
    }
	\end{tabular}
	\label{fig:furtherExpt1}
    \caption{Cumulative regret and choice of parameter $p$}
    \vspace*{-1em}
\end{figure}
%\vspace*{-0.5em}
\textbf{Third experiment (Large Horizon and uniform gaps):} This is conducted over a testbed of $20-100$ (interval of $10$) arms with Bernoulli reward distributions, where the expected rewards of the arms are $r_{i_{{i}\neq {*}}}=0.05$ and $r^{*}=0.1$. For each of these testbeds of $20-100$ arms, we report the cumulative regret over a large horizon $T=10^{5} + K_{20:100}^{3}$ timesteps averaged over $100$ independent runs. We report the performance of MOSS, OCUCB and EClusUCB only over this uniform gap setup. Algorithms like Thompson Sampling or Bayes-UCB are too slow to be run for such large $K$ (see \cite{lattimore2015optimally}). From the results in Figure \ref{fig:3}, it is evident that the growth of regret for EClusUCB is lower than that of OCUCB and nearly same as MOSS. This corroborates the finding of \cite{lattimore2015optimally} which states that MOSS breaks down only when the number of arms are exceptionally large or the horizon is unreasonably high and gaps are very small. We consistently see that in uniform gap testcases EClusUCB outperforms  OCUCB.

%The horizon $T$ is set to a very large value of $10^{5} + K^{3}$ timesteps and the number of arms are increased from $K=20$ to $100$.
%The proposed algorithm EClusUCB is run with same parameters as established in Corollary  \ref{Result:Corollary:2}. The regret is averaged over $100$ independent runs and is shown in Figure \ref{fig:3}. We report the performance of MOSS, OCUCB and EClusUCB only over this setup. From the results in Figure \ref{fig:3}, it is evident that the growth of regret for EClusUCB is lower than that of OCUCB and nearly same as MOSS. This corroborates the finding of \citet{lattimore2015optimally} as well, which says that MOSS breaks down only when the number of arms are exceptionally large or the horizon is unreasonably high and gaps are very small. This experiment also proves that our algorithm EClusUCB is stable for a large horizon and large number of arms.
%\vspace*{-0.5em}
\textbf{Fourth experiment (Choice of Cluster):} This is conducted to show that our choice of $p=\lceil\frac{K}{\log K}\rceil$ which we use to reduce the elimination error, is indeed close to optimal. The experiment is performed over a testbed having $30$ Bernoulli-distributed arms with $r_{i_{:{{i}\neq {*}}}}=0.07,\forall i\in A$ and $r^{*}=0.1$ averaged over 100 independent runs for each cluster. In Figure \ref{fig:4}, we report the cumulative regret over $T=80000$ timesteps averaged over $100$ independent runs plotted against the number of clusters $p=1$ to $\frac{K}{2}$ (so that each cluster have exactly two arms). We see that for $p=\lceil\frac{K}{\log K}\rceil=9$, the cumulative regret of EClusuCB is almost the lowest over the entire range of clusters considered. So, the choice of $p=\lceil\frac{K}{\log K}\rceil$ helps to balance both theoretical and empirical performance of EClusUCB. Also $p=1$ gives us EClusUCB-AE and we can clearly see that its cumulative regret is the highest among all the clusters considered showing clearly that clustering indeed has some benefits. Its poor performance stems from the fact that it eliminates optimal arm in many of the runs as opposed to EClusUCB. More experiments are shown in Appendix \ref{App:MoreExp}.

%The lowest is reached for $\frac{K}{2}=15$, but this would increase the elimination error of EClusUCB in our theoretical analysis.


%Also, we show cumulative regret for AClusUCB (which does not have $p$ as an input parameter) as a straight line, constant over the number of clusters. AClusUCB  performs poorly as compared to EClusUCB. We conjecture that this happens because AClusUCB conducts a significant amount of initial exploration to find the number of clusters or waits till the optimal arm settles in its own cluster which will eliminate all the other clusters by cluster elimination condition, as opposed to EClusUCB which has an uniform clustering scheme from the very start.



\section{Conclusions and future work}
\label{sec:conclusions}
From a theoretical viewpoint, we conclude that the gap-dependent regret bound of ClusUCB is lower than MOSS and UCB-Improved and its gap-independent regret bound is of the same order as MOSS ans OCUCB. From the numerical experiments on settings with small gaps between optimal and sub-optimal mean rewards, we observed that EClusUCB outperforms several popular bandit algorithms,  including OCUCB. Also ClusUCB is remarkably stable for a large horizon and large number of arms and performs well across different types of distributions. While we exhibited better regret bounds for ClusUCB, it would be interesting future research to improve the theoretical analysis of ClusUCB to achieve the gap-dependent regret bound of  OCUCB. This is also one of the first papers to apply clustering in stochastic MAB and another future direction is to use this in contextual or in distributed bandits. 

%Distributed bandits are a specific setup of MAB where a network of bandits collaborate with each other to identify the optimal arm(s) (see \cite{awerbuch2008competitive,liu2010distributed,szorenyi2013gossip,hillel2013distributed}). In our setting we can assign each of the $p$ clusters to individual bandits and at the end of each round they can share information synchronously to identify the optimal arm. This naturally results in a speedup of operation and helps in identifying the best arm faster. 


%\clearpage
%\newpage
% In the unusual situation where you want a paper to appear in the
% references without citing it in the main text, use \nocite
%\nocite{langley00}

\bibliography{biblio}
\bibliographystyle{apalike}
%\bibliographystyle{plain}


%%%%%%%%%%%%%%%%%%%%%%%%%%%%%%%%%%%%%%%%%%%%%%%%%%%%%%%%%%%%
%%%%%%%%%%%%%%%%%%%%%%%%%%%%%%%%%%%%%%%%%%%%%%%%%%%%%%%%%%%%

\clearpage
\newpage
%\onecolumn	
\section*{Appendix}

%\newpage
\appendix
The Appendix is organized as follows. First we prove some technical lemmas in Appendix \ref{App:Lemma1} and Appendix \ref{App:Lemma2}. Next we prove the main theorem in Appendix \ref{sec:proofTheorem}. In Appendix \ref{App:A} we prove Proposition \ref{proofTheorem:Prop:1}. In Appendix \ref{App:Proof:Corollary:1} we prove Corollary \ref{Result:Corollary:1}.

% and in Appendix \ref{App:Proof:Corollary:2} we prove Corollary \ref{Result:Corollary:2}. Appendix \ref{App:E} deals with the idea of why we do clustering. The simple regret bound of EClusUCB and its associated Corollary is proved in \ref{App:SR_EClusUCB}. Algorithm \ref{alg:aclusucb}, Adaptive Clustered UCB is shown in Appendix \ref{App:AClusUCB}. More experiments are shown in Appendix \ref{App:MoreExp}.

%\section{Regret Bound Table}
%\label{App:Table}
%\begin{table}[!h]
%\caption{Gap-dependent regret bounds for different bandit algorithms}
%\label{tab:regret-bds}
%\begin{center}
%\begin{tabular}{|c|c|}
%\toprule
%Algorithm  & Upper bound \\
%\midrule
%UCB1         &$O\left(\frac{K\log T}{\Delta}\right)$ \\\midrule
%UCB-Improved &$O\left(\frac{K\log (T\Delta^{2})}{\Delta}\right)$ \\\midrule
%MOSS	     &$O\left(\frac{K^{2}\log\left(T\Delta^{2}/K\right)}{\Delta}\right)$\\\midrule
%EClusUCB      &$O\left(\frac{K\log\left(\frac{T\Delta^{2}}{\sqrt{\log (K)}}\right)}{\Delta}\right)$\\\bottomrule
%\end{tabular}
%\end{center}
%\vspace*{-2em}
%\end{table}

\section{Proof of Lemma 1}
\label{App:Lemma1}

\begin{lemma}
\label{proofTheorem:Lemma:1}
If $T\geq K^{2.4}$, $\psi=\dfrac{T}{ K^2}$, $\rho_a =\dfrac{1}{2}$ and $m\leq \dfrac{1}{2} \log_2\left(\dfrac{T}{e}\right) $, then,
\begin{align*}
\dfrac{\rho_a m \log(2)}{\log(\psi T) - 2m\log( 2)} \leq \frac{3}{2}
\end{align*}
\end{lemma}

\begin{proof}
The proof is based on contradiction. Suppose
\begin{eqnarray*}
\dfrac{\rho m \log(2)}{\log(\psi T) - 2m\log( 2)} > \frac{3}{2}.
\end{eqnarray*}
Then, with $\psi=\dfrac{T}{ K^2}$ and $\rho_a =\dfrac{1}{2}$, we obtain
\begin{eqnarray*}
6\log(K) 
&>& 6\log(T) - 7m\log(2) \\
&\overset{(a)}{\ge}& 6\log(T) - \frac{7}{2} \log_2\left(\frac{T}{e}\right) \log(2) \\
&=& 2.5\log(T) + 3.5 \log_2(e)\log(2)  \\
&\overset{(b)}{=}& 2.5\log(T) +3.5
\end{eqnarray*}
where $(a)$ is obtained using $m\leq \dfrac{1}{2} \log_2\left(\dfrac{T}{e}\right)$, while $(b)$ follows from the identity $\log_2(e)\log(2) =1$. Finally, for $T\ge K^{2.4}$ we obtain, $6\log(K)>6\log(K)+3.5$, which is a contradiction. Hence, for $T\geq K^{2.4}$, $\psi=\dfrac{T}{ K^2}$, $\rho=\dfrac{1}{2}$ and $m \leq \dfrac{1}{2} \log_2\left(\dfrac{T}{e}\right) $ we have, 
\begin{align*}
\dfrac{\rho m \log(2)}{\log(\psi T) - 2m\log( 2)} \leq \frac{3}{2}
\end{align*}
\end{proof}

\section{Proof of Lemma 2}
\label{App:Lemma2}
\begin{lemma}
\label{proofTheorem:Lemma:2}
If $T\geq K^{2.4}$, $\psi=\dfrac{T}{ K^2}$, $\rho_a =\dfrac{1}{2}$, $m_i = min\lbrace m|\sqrt{2\epsilon_{m} } < \dfrac{\Delta_i}{4} \rbrace $ and $c_{m_i} =\sqrt{\frac{\rho_{a}\log (\psi T\epsilon_{m_{i}})}{2 n_{m_i}}}$, then, $c_{m_i} < \dfrac{\Delta_i}{4}$.
\end{lemma}

\begin{proof}
In the $m_i$-th round $c_{m_i} =\sqrt{\frac{\rho_{a}\log (\psi T\epsilon_{m_{i}})}{2 n_{m_i}}}$. Substituting the value of $n_{m_i}=\dfrac{\log{(\psi T\epsilon_{m_{i}}^{2})}}{2\epsilon_{m_{i}}}$ in $c_{m_i}$ we get,
\begin{align*}
	c_{m_i} &\leq \sqrt{\dfrac{\rho_a \epsilon_{m_{i}}\log (\psi T\epsilon_{m_{i}})}{\log(\psi T\epsilon_{m_{i}}^{2})}} \leq \sqrt{\dfrac{\rho_a \epsilon_{m_{i}}\log (\frac{\psi T\epsilon_{m_{i}}^{2}}{\epsilon_{m_{i}}})}{\log(\psi T\epsilon_{m_{i}}^{2})}} \\
	%%%%%%%%%%%%%%%%%%%%%%%%%%
	& = \sqrt{\dfrac{\rho_a\epsilon_{m_{i}}\log (\psi T\epsilon_{m_{i}}^{2}) - \rho_a\epsilon_{m_{i}}\log (\epsilon_{m_{i}})}{\log(\psi T\epsilon_{m_{i}}^{2})}} 
	\leq  \sqrt{\rho_a\epsilon_{m_{i}} - \dfrac{\rho_a\epsilon_{m_i}\log(\frac{1}{2^{m_i}})}{\log(\psi T \frac{1}{2^{2m_i}})}} \\
	%%%%%%%%%%%%%%%%%%%%%%%%%%
	&\leq \sqrt{\rho_a\epsilon_{m_{i}} + \dfrac{\rho_a\epsilon_{m_i}\log(2^{m_i})}{\log(\psi T) - \log( 2^{2m_i})}}  \leq \sqrt{\rho_a\epsilon_{m_{i}} + \dfrac{\rho_a\epsilon_{m_i}m_i \log(2)}{\log(\psi T) - 2m_i\log( 2)}} \\ 
	%%%%%%%%%%%%%%%%%%%%%%%%%%
	 & \overset{(a)}{\leq} \sqrt{\rho_a\epsilon_{m_{i}} + \frac{3}{2}\epsilon_{m_i}} 
	  < \sqrt{2\epsilon_{m_i}} 
	  < \dfrac{\Delta_{i}}{4} 
	\end{align*}
In the above simplification, $(a)$ is obtained using Lemma~\ref{proofTheorem:Lemma:1}. 
\end{proof}



\section{Proof of Theorem 1}
\label{sec:proofTheorem}
\begin{proof}
Let $A^{'}=\lbrace i \in A,\Delta_{i}> b\rbrace$,  $A^{''}=\lbrace i \in A, \Delta_{i} > 0\rbrace$, $A^{'}_{s_{k}}=\lbrace i \in A_{s_{k}},\Delta_{i}> b\rbrace$ and $A^{''}_{s_{k}}=\lbrace i \in A_{s_{k}}, \Delta_{i} > 0 \rbrace$. $C_{g}$ is the cluster set containing max payoff arm from each cluster in $g$-th round. The arm having the true highest payoff in a cluster $s_{k}$ is denote by $a_{\max_{s_{k}}}$. Let for each sub-optimal arm ${i}\in A$, $m_{i}=\min{\lbrace m|\sqrt{2\epsilon_{m}} < \frac{\Delta_{i}}{4} \rbrace}$ and let for each cluster $s_{k}\in S$, $g_{s_{k}}=\min{\lbrace g|\sqrt{2\epsilon_{g}} < \frac{\Delta_{a_{\max_{s_{k}}}}}{4} \rbrace}$. Let $\check{A}=\lbrace {i}\in A^{'} | {i}\in s_{k} , \forall s_{k}\in S \rbrace$. The analysis proceeds by considering the contribution to the regret in each of the following cases:

\textbf{Case a:} \textit{Some sub-optimal arm ${i}$ is not eliminated in round $\max(m_{i},g_{s_{k}})$ or before, with the optimal arm ${*}\in C_{\max(m_{i},g_{s_{k}})}$.}
We consider an arbitrary sub-optimal arm ${i}$ and analyze the contribution to the regret when $i$ is not eliminated in the following exhaustive sub-cases:\\
\textbf{Case a1:} \textit{In round $\max(m_{i},g_{s_{k}})$, ${i} \in s^{*}$.}

Similar to case (a) of \cite{auer2010ucb}, observe that when the following two conditions hold, arm $i$ gets eliminated:
\begin{align}
\hat{r}_{i}  \le r_{i} + c_{m_i} \text{ and } 
 \hat{r}^{*}\geq  r^{*} - c_{m_i}, \label{eq:armelim-casea1}
\end{align}
where  $c_{m_i} = \sqrt{\frac{\rho_{a}\log (\psi T\epsilon_{m_{i}})}{2 n_{m_i}}}$. The arm $i$ gets eliminated because 
  \begin{align*}
\hat{r}_{i} +c_{m_i} & \leq r_{i} + 2c_{m_i} < r_{i} + \Delta_{i} - 2c_{m_i}\\
 &\leq r^{*} -2c_{m_i} \leq \hat{r}^{*} - c_{m_i}  .
  \end{align*}
In the above, we have used the fact that $ c_{m_i} = \sqrt{\epsilon_{m_{i}+1}} < \frac{\Delta_{i}}{4}$,  from Lemma~\ref{proofTheorem:Lemma:2}. From the foregoing, we have to bound the events complementary to that in \eqref{eq:armelim-casea1} for an arm $i$ to not get eliminated. Considering Chernoff-Hoeffding bound this is done as follows:
  \begin{align*}
&\mathbb{P}\left(\hat{r}_{i}\geq r_{i} + c_{m_i}\right)\leq \exp(-2c_{m_i}^{2}n_{m_i})\\
&\leq \exp(-2 * \frac{\rho_{a}\log (\psi T\epsilon_{m_{i}})}{2 n_{m_i}} *n_{m_i})
\leq \frac{1}{(\psi T\epsilon_{m_{i}})^{\rho_{a}}}   
  \end{align*}
Along similar lines, we have $\mathbb{P}\left(\hat{r}^{*}\leq r^{*} - c_{m_i}\right)\leq \frac{1}{(\psi  T\epsilon_{m_{i}})^{\rho_{a}}}.$ Thus, the probability that a sub-optimal arm ${i}$ is not eliminated in any round on or before $m_{i}$ is bounded above by $\bigg(\frac{2}{(\psi T\epsilon_{m_{i}})^{\rho_{a}}}\bigg)$. 
 Summing up over all arms in $A_{s^{*}}^{'}$ in conjunction with a simple bound of $T\Delta_{i}$ for each arm we obtain,
   \begin{align*}
&\sum_{i\in A_{s^{*}}^{'}}\left(\dfrac{2T\Delta_{i}}{(\psi T\epsilon_{m_{i}}^{2})^{\rho_{a}}}\right)
\leq\sum_{i\in A_{s^{*}}^{'}}\left(\frac{2T\Delta_{i}}{(\psi T\dfrac{\Delta_{i}^{4}}{16})^{\rho_{a}}}\right)
\overset{(a)}{\leq} \sum_{i\in A_{s^{*}}^{'}}\left(\frac{2T\Delta_{i}}{(\dfrac{T^2}{K^2}\dfrac{\Delta_{i}^{2}}{32})^{\frac{1}{2}}}\right)\leq 8\sqrt{2} \sum_{i\in A_{s^{*}}^{'}}K
%\leq \sum_{i\in A_{s^{*}}^{'}}\bigg(\frac{C_{1}(\rho_{a})T^{1-\rho_{a}}}{\Delta_{i}^{4\rho_{a}-1}}\bigg) %\text{, where } C_1(x) = \frac{2^{1+4x}}{\psi^{x}}
   \end{align*}
   
  Here, in $(a)$ we substituted the value $\rho_a$ and $\psi$.

%%%%%%%%%%%%%%%%%%%%%%%%%%%%%%%%%%%%%%%%%%%%%%%%%%%%%%%%%%%%%%%%%%%%%%%%%%%%%%%%%%%%%%%%%%%%%%%   



%%%%%%%%%%%%%%%%%%%%%%%%%%%%%%%%%%%%%%%%%%%%%%%%%%%%%%%%%%%%%%%%%%%%%%%%%%%%%%%%%%%%%%%%%%%%%%%   
\textbf{Case a2:} \textit{In round $\max(m_{i},g_{s_{k}})$, ${i} \in s_k$ for some $s_k \ne s^{*}$.}

Following a parallel argument like in Case $a1$, we have to bound the following two events of arm $a_{\max_{s_{k}}}$ not getting eliminated on or before $g_{s_{k}}$-th round,
\begin{align*}
  \hat{r}_{a_{\max_{s_{k}}}} \geq r_{a_{\max_{s_{k}}}} +c_{g_{s_{k}}} \text{ and } \hat{r}^{*} \leq r^{*} - c_{g_{s_{k}}} 
\end{align*} 

We can prove using Chernoff-Hoeffding bounds and considering independence of events mentioned above, that for $c_{g_{s_{k}}}=\sqrt{\frac{\rho_{s} \log (\psi T\epsilon_{g_{s_{k}}})}{2 n_{g_{s_{k}}}}}$ and  $n_{g_{s_{k}}}=\frac{\log{(\psi T\epsilon_{g_{s_{k}}}^{2})}}{2\epsilon_{g_{s_{k}}}}$ the probability of the above two events is bounded by $\bigg(\dfrac{2}{(\psi  T\epsilon_{g_{s_{k}}})^{\rho_{s}}}\bigg)$.

  Now, for any round $g_{s_{k}}$, all the elements of $C_{\max(m_{i},g_{s_{k}})}$ are the respective maximum payoff arms of their cluster $s_{k}, \forall s_{k}\in S$, and since clusters are fixed so we can bound the maximum probability that a sub-optimal arm ${i}\in A^{'}$  and ${i}\in s_{k}$ such that $a_{\max_{s_{k}}}\in C_{g_{s_{k}}}$ is not eliminated on or before the $g_{s_{k}}$-th round by the same probability as above. Summing up over all $p$ clusters and bounding the regret for each arm $i\in A_{s_{k}}^{'}$ trivially by $T\Delta_{i}$,
 \begin{align*}
 &\sum_{k=1}^{p}\sum_{i\in A_{s_{k}}^{'}}\left(\frac{2T\Delta_{i}}{(\psi T\frac{\Delta_{i}^{2}}{16})^{\rho_{s}}}\right) = \sum_{i\in A^{'}}\bigg( \frac{2T\Delta_{i}}{(\psi  T\frac{\Delta_{i}^{2}}{16})^{\rho_{s}}} \bigg) \\
 & \overset{(a)}{\leq} \sum_{i\in A^{'}}\left(\frac{2T\Delta_{i}}{(\frac{T^2}{K^2}\frac{\Delta_{i}^{2}}{32})^{\frac{1}{2}}}\right) = \sum_{i\in A^{'}} \left( 8\sqrt{2}K \right)
% &\leq \sum_{i\in A^{'}}\bigg(\frac{2^{1+4\rho_{s}}T^{1-\rho_{s}}}{\psi^{\rho_{s}}\Delta_{i}^{4\rho_{s}-1}}\bigg) = \sum_{i\in A^{'}}\frac{C_{1}(\rho_{s})T^{1-\rho_{s}}}{\Delta_{i}^{4\rho_{s}-1}}
 \end{align*}

Again we obtain $(a)$ by substituting the value of $\rho_s$ and $\psi$.

Summing the bounds in Cases $a1-a2$ and observing that the bounds in the aforementioned cases hold for any round $C_{\max \lbrace m_i,g_{s_k}\rbrace}$, we obtain the following contribution to the expected regret from case a:
   %Taking summation of the events mentioned above($a1$-$a4$) gives us an upper bound on the regret given that the optimal arm $a^{*}$ is still surviving, 
\begin{align*}
& \sum_{i\in A_{s^*}^{'}} 8\sqrt{2} K + \sum_{i\in A^{'}} 8\sqrt{2} K \leq \sum_{i\in A_{s^*}^{'}} 12 K + \sum_{i\in A^{'}} 12 K
%&\sum_{i\in A_{s^*}} \frac{C_{1}(\rho_{a})T^{1-\rho_{a}}}{\Delta_{i}^{4\rho_{a}-1}} + \sum_{i\in A^{'}}\bigg(\frac{C_{1}(\rho_{s})T^{1-\rho_{s}}}{\Delta_{i}^{4\rho_{s}-1}}\bigg)
\end{align*}

%%%%%%%%%%%%%%%%%%%%%%%%%%%%%%%%%%%%%%%%%%%%%%%%%%%%%%%%%%%%%%%%%%%%%%%%%%%%%%%%%%%%%%%%%%%%
\textbf{Case b:} \textit{For each arm $i$, either ${i}$ is eliminated in round $\max (m_{i},g_{s_{k}})$ or before or there is no optimal arm ${*}$ in $C_{\max(m_{i},g_{s_{k}})}$.} \\
\textbf{Case b1:} \textit{${*}\in C_{\max(m_{i},g_{s_{k}})}$ for each arm $i \in A'$ and cluster $s_k \in \check A$.} 
The condition in the case description above implies the following: \\
\begin{inparaenum}[\bfseries (i)]
\item each sub-optimal arm ${i}\in A^{'}$ is  eliminated on or before $\max (m_{i},g_{s_{k}})$ and hence  pulled not more than $ n_{m_{i}}$ number of times.\\
\item each sub-optimal cluster $s_k \in \check A$ is  eliminated on or before $\max (m_{i},g_{s_{k}})$ and hence  pulled not more than $ n_{g_{s_{k}}}$ number of times.
\end{inparaenum}

Hence, the maximum regret suffered due to pulling of a sub-optimal arm or a sub-optimal cluster is no more than the following:
 \begin{align*}
 &\sum_{i\in A^{'}}\Delta_{i}\bigg\lceil\dfrac{\log{(\psi T\epsilon_{m_{i}}^{2})}}{2\epsilon_{m_{i}}}\bigg\rceil 
\!+\! \sum_{k=1}^{p}\sum_{i\in A_{s_{k}}^{'}}\Delta_{i}\bigg\lceil\dfrac{\log{(\psi T\epsilon_{g_{s_{k}}}^{2})}}{2\epsilon_{g_{s_{k}}}}\bigg\rceil \\
%%%%%%%%%%%%%%%%%%%%
&\overset{a}{\leq}\sum_{i\in A^{'}}\Delta_{i}\bigg(1+\dfrac{16\log{\left(\psi T\left(\frac{\Delta_{i}}{2}\right)^{4}\right)}}{\Delta_{i}^{2}}\bigg) 
\quad+ \sum_{i\in A^{'}}\Delta_{i}\bigg(1+\dfrac{16\log{\left(\psi T\left(\frac{\Delta_{i}}{2}\right)^{4}\right)}}{\Delta_{i}^{2}}\bigg)
\\
%%%%%%%%%%%%%%%%%%%%
 &\overset{b}{\leq} \sum_{i\in A^{'}}\!\bigg[ 2\Delta_{i}+\dfrac{16(\log{(\frac{T^2}{K^2}\frac{\Delta_{i}^{4}}{1024})} + \log{(\frac{T^2}{K^2}\frac{\Delta_{i}^{4}}{1024})})}{\Delta_{i}} \bigg] \leq \sum_{i\in A^{'}}\!\bigg[ 2\Delta_{i}+\dfrac{32\left(\log{(\frac{T\Delta_{i}^2}{K})} + \log{(\frac{T\Delta_{i}^2}{K})}\right)}{\Delta_{i}} \bigg]
%  \\
% & \qquad \qquad +\dfrac{32\rho_{s}\log{(\psi T\dfrac{\Delta_{i}^{4}}{16\rho_{s}^{2}})}}{\Delta_{i}}\bigg\rbrace 
 \end{align*}
In the above, the $(a)$ follows since $\sqrt{2\epsilon_{m_{i}}} < \frac{\Delta_{i}}{4}$ and $\sqrt{2\epsilon_{n_{g_{s_{k}}}}} < \frac{\Delta_{a_{\max_{s_{k}}}}}{4}$ and $(b)$ is obtained by substituting the values of $\rho_a,\rho_s$ and $\psi$.

%&\leq\Delta_{i}\bigg(1+\dfrac{32\rho_{a}\log{(\psi T\dfrac{\Delta_{i}^{4}}{16\rho_{a}^{2}})}}{\Delta_{i}^{2}}\bigg)\\
%&\leq\Delta_{i}\bigg\lceil\dfrac{2\log{(\psi T(\dfrac{\Delta_{i}}{2\sqrt{\rho_{a}})^{4})}}}{(\dfrac{\Delta_{i}}{2\sqrt{\rho_{a}}})^{2}}\bigg\rceil \\
%\text{, since } \sqrt{\rho_{a}\epsilon_{m_{i}}}\leq\dfrac{\Delta_{i}}{2}
 
%%%%%%%%%%%%%%%%%%%%%%%%%%%%%%%%%%%%%%%%%%%%%%%%%%%%%%%%%%%%%%%%%%%%%%%%%%%%%%%%%%%%%%%%%%%%%%%   
%\textbf{Case b2:} \textit{Optimal arm $a^{*}$ is eliminated by a sub-optimal arm.}\\
  %
	%This, can happen in $3$ ways,
%\newline
\textbf{Case b2:} \textit{${*}$ is eliminated by some sub-optimal arm in $s^*$} \\
%In this case, we are concerned with the arm elimination condition only. 
Optimal arm $*$ can get eliminated by some sub-optimal arm $i$ only if arm elimination condition holds, i.e., 
\begin{align*}
\hat r_{i} - c_{m_i} > \hat{r}^{*}+ c_{m_i},
\end{align*}
where, as mentioned before, $c_{m_i}  =\sqrt{\frac{\rho_{a}\log (\psi T\epsilon_{m_{i}})}{2 n_{m_i}}}$.
From analysis in Case $a1$, notice that, if \eqref{eq:armelim-casea1} holds in conjunction with the above, arm $i$ gets eliminated. Also, recall from Case $a1$ that the events complementary to \eqref{eq:armelim-casea1} have low-probability and can be upper bounded by $\frac{2}{(\psi  T\epsilon_{m_{*}})^{\rho_{a}}}$. Moreover, a sub-optimal arm that eliminates $*$ has to survive until round $m_*$. In other words, all arms ${j}\in s^{*}$ such that $m_{j} < m_{*}$ are eliminated on or before $m_*$ (this corresponds to case b1). Let, the arms surviving till $m_{*}$ round be denoted by $A^{'}_{s^{*}}$. This leaves any arm $a_{b}$ such that $m_{b}\geq m_{*} $ to still survive and eliminate arm ${*}$ in round $m_{*}$. Let, such arms that survive ${*}$ belong to $A^{''}_{s^{*}}$. Also maximal regret per step after eliminating ${*}$ is the maximal $\Delta_{j}$ among the remaining arms in $A^{''}_{s^{*}}$ with $m_{j}\geq m_{*}$.  Let $m_{b}=\min\lbrace m|\sqrt{2\epsilon_{m}}<\frac{\Delta_{b}}{4}\rbrace$. Hence, the maximal regret after eliminating the arm ${*}$ is upper bounded by, 
%Let $C_2(x) = \frac{2^{2x+\frac{3}{2}}}{\psi^{x}}$.
\begin{align*}
&\sum_{m_{*}=0}^{max_{j\in A^{'}_{s^{*}}}m_{j}}\sum_{\substack{i\in A^{''}_{s^{*}}: \\ m_{i}\geq m_{*}}}\bigg(\dfrac{2}{(\psi  T\epsilon_{m_{*}})^{\rho_{a}}} \bigg).T\max_{\substack{j\in A^{''}_{s^{*}}: \\ m_{j}\geq m_{*}}}{\Delta}_{j}\\
%%%%%%%%%%%%%%%
&\leq\sum_{m_{*}=0}^{max_{j\in A^{'}_{s^{*}}}m_{j}}\sum_{i\in A^{''}_{s^{*}}:m_{i} \geq m_{*}}\bigg(\dfrac{2}{(\psi  T\epsilon_{m_{*}})^{\rho_{a}}} \bigg).T.4\sqrt{2\epsilon_{m_{*}}} \\
%%%%%%%%%%%%%%%
&\leq\sum_{m_{*}=0}^{max_{j\in A^{'}_{s^{*}}}m_{j}}\sum_{i\in A^{''}_{s^{*}}:m_{i} \geq m_{*}}8\sqrt{2}\bigg(\dfrac{T^{1-\rho_{a}}}{\psi^{\rho_{a}}\epsilon_{m_{*}}^{\rho_{a}-\frac{1}{2}}} \bigg)\\
%%%%%%%%%%%%%%%
&\leq\sum_{i\in A^{''}_{s^{*}}:m_{i} \geq m_{*}}\sum_{m_{*}=0}^{\min{\lbrace m_{i},m_{b}\rbrace}}\bigg(\dfrac{8\sqrt{2} T^{1-\rho_{a}}}{\psi^{\rho_{a}}2^{-(\rho_{a}-\frac{1}{2})m_{*}}} \bigg)\\
%%%%%%%%%%%%%%
&\!\leq\!\!\sum_{i\in A^{'}_{s^{*}}}\frac{8\sqrt{2} T^{1-\rho_{a}}}{\psi^{\rho_{a}}2^{-(\rho_{a}-\frac{1}{2})m_{*}}}\! +\!\!\!\sum_{i\in A^{''}_{s^{*}}\setminus A^{'}_{s^{*}}}\!\frac{8\sqrt{2} T^{1-\rho_{a}}}{\psi^{\rho_{a}}2^{-(\rho_{a}-\frac{1}{2})m_{b}}} \\
%%%%%%%%%%%%%%
&\!\leq\!\!\sum_{i\in A^{'}_{s^{*}}}\frac{T^{1-\rho_{a}}2^{\rho_{a}+\frac{7}{2}}}{\psi^{\rho_{a}}\Delta_{i}^{2\rho_{a}-1}} \!+\!\!\!\sum_{i\in A^{''}_{s^{*}}\setminus A^{'}_{s^{*}}}\!\!\frac{T^{1-\rho_{a}}2^{\rho_{a}+\frac{7}{2}}}{\psi^{\rho_{a}}b^{2\rho_{a}-1}} \\
%%%%%%%%%%%%%%
& \leq \sum_{i\in A^{'}_{s^{*}}} 16K \!+\!\!\!\sum_{i\in A^{''}_{s^{*}}\setminus A^{'}_{s^{*}}}\!\! 16K
%& = \sum_{i\in A^{'}_{s^{*}}}\dfrac{ C_{2}(\rho_{a}) T^{1-\rho_{a}}}{\Delta_{i}^{4\rho_{a}-1}} +\sum_{i\in A^{''}_{s^{*}}\setminus A^{'}_{s^{*}}}\dfrac{C_{2(\rho_{a})}T^{1-\rho_{a}}}{b^{4\rho_{a}-1}}.
\end{align*}

%%%%%%%%%%%%%%%%%%%%%%%%%%%%%%%%%%%%%%%%%%%%%%%%%%%%%%%%%%%%%%%%%%%%%%%%%%%%%%%%%%%%%%%%%%%%%%%

%%%%%%%%%%%%%%%%%%%%%%%%%%%%%%%%%%%%%%%%%%%%%%%%%%%%%%%%%%%%%%%%%%%%%%%%%%%%%%%%%%%%%%%%%%%%%%%   
\textbf{Case b3:} \textit{$s^{*}$ is eliminated by some sub-optimal cluster.} 
Let $C_{g}^{'}=\lbrace a_{max_{s_{k}}}\in A^{'}|\forall s_{k}\in S \rbrace$ and $C_{g}^{''}=\lbrace a_{max_{s_{k}}}\in A^{''}|\forall s_{k}\in S \rbrace$. A sub-optimal cluster $s_k$ will eliminate $s^*$ in round $g_*$ only if the cluster elimination condition of Algorithm \ref{alg:clusucb} holds, which is the following when ${*}\in C_{g_{*}}$:
\begin{align}
\hat r_{a_{\max_{s_k}}} - c_{g_*} > \hat{r}^{*}+ c_{g_*}.
\label{eq:caseb3-cluselim}
\end{align}
Notice that when ${*}\notin C_{g_{*}}$, since $r_{a_{max_{s_{k}}}}>r^{*}$, the inequality in \eqref{eq:caseb3-cluselim} has to hold for cluster $s_k$ to eliminate $s^*$.
As in case $b2$, the probability that a given sub-optimal cluster $s_k$ eliminates $s^*$ is upper bounded by  $\frac{2}{(\psi T\epsilon_{g_{s^{*}}})^{\rho_{s}}}$ and all sub-optimal clusters with $g_{s_{j}}< g_{*}$ are eliminated before round $g_*$. This leaves any arm $a_{\max_{s_{b}}}$ such that $g_{s_{b}}\geq g_{*}$ to still survive and eliminate arm ${*}$ in round $g_{*}$. Let, such arms that survive ${*}$ belong to $C_{g}^{''}$. Hence, following the same way as case $b2$,  the maximal regret after eliminating ${*}$ is,
 \begin{align*}
 \!\!\sum_{g_{*}=0}^{\max\limits_{a_{\max_{s_{j}}}\in C_{g}^{'}}g_{s_{j}}}\!\!\!\!\!\sum_{\substack{\scriptsize a_{\max_{s_{k}}}\in C_{g}^{''}: \\ g_{s_{k}} \geq g_{*}}}\bigg(\dfrac{2}{(\psi T\epsilon_{g_{s^{*}}})^{\rho_{s}}} \bigg)T\max_{\substack{a_{\max_{s_{j}}}\in C_{g}^{''}: \\ g_{s_{j}}\geq g_{*}}}{\Delta}_{a_{\max_{s_{j}}}}
 \end{align*}
Using $A'\supset C_{g}^{'}$ and $A''\supset C_{g}^{''}$, we can bound the regret contribution from this case in a similar manner as Case $b2$ as follows:
\begin{align*}
 &\!\!\sum_{i\in A^{'}\setminus A_{s^*}^{'}}\frac{T^{1-\rho_{s}}2^{\rho_{s}+\frac{5}{2}}}{\psi^{\rho_{s}}\Delta_{i}^{2\rho_{s}-1}} 
 \!+\!\!\!\sum_{i\in A^{''}\setminus A^{'}\cup A_{s^*}^{'}}\!\!\!\!\frac{T^{1-\rho_{s}}2^{\rho_{s}+\frac{5}{2}}}{\psi^{\rho_{s}}b^{2\rho_{s}-1}} \\
 & = \sum_{i\in A^{'}\setminus A_{s^*}^{'}} 16K +\sum_{i\in A^{''}\setminus A^{'}\cup A_{s^*}^{'}} 16K
% & = \sum_{i\in A^{'}\setminus A_{s^*}^{'}}\frac{C_{2}(\rho_{s})T^{1-\rho_{s}}}{\Delta_{i}^{4\rho_{s}-1}} +\sum_{i\in A^{''}\setminus A^{'}\cup A_{s^*}^{'}}\frac{C_{2}(\rho_{s})T^{1-\rho_{s}}}{b^{4\rho_{s}-1}} 
\end{align*}

%%%%%%%%%%%%%%%%%%%%%%%%%%%%%%%%%%%%%%%%%%%%%%%%%%%%%%%%%%%%%%%%%%%%%%%%%%%%%%%%%%%%%%%%%
\textbf{Case b4:} \textit{${*}$ is not in $C_{\max(m_{i},g_{s_{k}})}$, but belongs to $B_{\max(m_{i},g_{s_{k}})}$.}

In this case the optimal arm ${*}\in s^{*}$ is not eliminated, also $s^{*}$ is not eliminated. So, for all sub-optimal arms $i$ in $A_{s^*}^{'}$ which gets eliminated on or before $\max \lbrace m_{i},g_{s_{k}} \rbrace$ will get pulled no more than $ \left\lceil\dfrac{\log{(\psi T\epsilon_{m_{i}}^{2})}}{2\epsilon_{m_{i}}}\right\rceil$ number of times, which leads to the following bound the contribution to the expected regret, as in Case $b1$:
\begin{align*}
 &\sum_{i\in A_{s^*}^{'}}\bigg\lbrace \Delta_{i}+\dfrac{32\log{(\frac{T\Delta_i^2}{K})}}{\Delta_{i}} \bigg\rbrace 
\end{align*} 

For arms $a_i \notin s^*$, the contribution to the regret cannot be greater than that in Case $b3$. So the regret is bounded by,

\begin{align*}
\sum_{i\in A^{'}\setminus A_{s^*}^{'}} 16K +\sum_{i\in A^{''}\setminus A^{'} \cup A_{s^*}^{'}} 16K
%\sum_{i\in A^{'}\setminus A_{s^*}^{'}}\dfrac{C_{2}(\rho_{s})T^{1-\rho_{s}}}{\Delta_{i}^{4\rho_{s}-1}} +\sum_{i\in A^{''}\setminus A^{'} \cup A_{s^*}^{'}}\dfrac{C_{2}(\rho_{s})T^{1-\rho_{s}}}{b^{4\rho_{s}-1}}
\end{align*}
The main claim follows by summing the contributions to the expected regret from each of the cases above.
\end{proof}





\section{Proof of Proposition 1}
\label{App:A}

\begin{proof}
Let $p=1$ such that all the arms in $A$ belongs to a single cluster. Hence, in ClusUCB-AE there is only arm elimination and no cluster elimination. Let, for each sub-optimal arm ${i}$, $m_{i}=\min{\lbrace m|\sqrt{\epsilon_{m}} < \dfrac{\Delta_{i}}{2} \rbrace}$. Also $\rho_{a}=\frac{1}{2}$ is a constant in this proof. Let $A^{'}=\lbrace i\in A: \Delta_{i} > b \rbrace$ and $A^{''}=\lbrace i\in A: \Delta_{i} > 0 \rbrace$. 

%Also $z_{i}$ denotes total number of times an arm $i$ has been pulled. In the $m$-th round, $n_{m}$ denotes the number of pulls allocated to the surviving arms in $B_{m}$. 
%The theoretical analysis remains same as we have always bounded the values of $\rho_{a}\in (0,1]$.

\subsection*{Case $a$: \textit{Some sub-optimal arm ${i}$ is not eliminated in round $m_{i}$ or before and the optimal arm ${*}\in B_{m_{i}}$}}
  
	Following the steps of Theorem \ref{Result:Theorem:1} Case $a1$, an arbitrary sub-optimal arm ${i}\in A^{'}$ can get eliminated only when the event,
	\begin{align}
	\hat{r}_{i}  \le r_{i} + c_{m_i} \text{ and } \label{eq:appA:armelim-casea}
 	\hat{r}^{*}\geq  r^{*} - c_{m_i}
	\end{align}
	
	takes place. So to bound the regret we need to bound the probability of the complementary event of these two conditions. Note that  $c_{m_{i}} = \sqrt{\frac{\rho_{a}\log (\psi T\epsilon_{m_{i}})}{2 n_{m_i}}}$. A sub-optimal arm $i$ will get eliminated in the $m_i$-th round because $n_{m_{i}}=\dfrac{\log{(\psi T\epsilon_{m_{i}}^{2})}}{2\epsilon_{m_{i}}}$ and substituting this in $c_{m_i}$ and applying Lemma \ref{proofTheorem:Lemma:2} we get, $c_{m_i} < \dfrac{\Delta_{i}}{4} $.

  Again, for ${i} \in A^{'}$, 
  \begin{align*}
\hat{r}_{i} + c_{m_i}&\leq r_{i} + 2c_{m_i} 
%&= \hat{r}_{i} + 4c_{m_{i}} - 2c_{m_{i}} \\
 < r_{i} + \Delta_{i} - 2c_{m_i}
 \leq r^{*} -2c_{m_i} 
 \leq \hat{r}^{*} - c_{m_i}
  \end{align*}

	Applying Chernoff-Hoeffding bound and considering independence of complementary of the two events in \ref{eq:appA:armelim-casea},
  \begin{align*}
\mathbb{P}\lbrace\hat{r}_{i}&\geq r_{i} + c_{m_i}\rbrace\leq \exp(-2c_{m_i}^{2}n_{m_{i}})
\leq \exp(-2 * \dfrac{\rho_{a}\log (\psi T\epsilon_{m_{i}})}{2 n_{m_{i}}} *n_{m_{i}})
\leq \dfrac{1}{(\psi T\epsilon_{m_{i}})^{\rho_{a}}}   
  \end{align*}
 
%$\leq \bigg(\dfrac{1}{4\psi T\epsilon_{m}^{2}}\bigg)^{D}$, as $\ell_{m}-1\leq D$
% \hspace*{2em}
 
Similarly, $\mathbb{P}\lbrace\hat{r}^{*}\leq r^{*} - c_{m_i}\rbrace\leq \dfrac{1}{(\psi  T\epsilon_{m_{i}})^{\rho_{a}}}$. Summing the two up, the probability that a sub-optimal arm ${i}$ is not eliminated on or before $m_{i}$-th round is  $\bigg(\dfrac{2}{(\psi T\epsilon_{m_{i}})^{\rho_{a}}}\bigg)$. 
 
Summing up over all arms in $A^{'}$ and bounding the regret for each arm $i\in A^{'}$ trivially by $T\Delta_{i}$, we obtain
   \begin{align*}
\sum_{i\in A^{'}}\bigg(\dfrac{2T\Delta_{i}}{(\psi T\epsilon_{m_{i}})^{\rho_{a}}}\bigg)
\leq\sum_{i\in A^{'}}\bigg(\dfrac{2T\Delta_{i}}{(\psi T\dfrac{\Delta_{i}^{2}}{32})^{\rho_{a}}}\bigg)
&\leq \sum_{i\in A^{'}}\bigg(\dfrac{2^{1+4\rho_{a}}T^{1-\rho_{a}}\Delta_{i}}{\psi^{\rho_{a}}\Delta_{i}^{2\rho_{a}}}\bigg)
\leq \sum_{i\in A^{'}}\bigg(\dfrac{2^{1+5\rho_{a}}T^{1-\rho_{a}}}{\psi^{\rho_{a}}\Delta_{i}^{2\rho_{a}-1}}\bigg)\\  
%%%%%%%%%%%%%%%%%% 
& \overset{(a)}{\leq}\sum_{i\in A^{'}}\leq 8\sqrt{2} K
   \end{align*}

Here, $(a)$ is obtained by substituting the values of $\psi$ and $\rho_a$.

\subsection*{Case $b$: \textit{Either an arm ${i}$ is eliminated in round $m_{i}$ or before or else there is no optimal arm ${*}\in B_{m_{i}}$ }}

\subsubsection*{Case $b1$: \textit{${*}\in B_{m_{i}}$ and each ${i}\in A^{'}$ is  eliminated on or before $m_{i}$ } }

 Since we are eliminating a sub-optimal arm ${i}$ on or before round $m_{i}$, it is pulled no longer than, 
 \begin{align*}
  \bigg\lceil\dfrac{\log{(\psi T\epsilon_{m_{i}}^{2})}}{2\epsilon_{m_{i}}}\bigg\rceil
 \end{align*}

So, the total contribution of ${i}$  till round $m_{i}$ is given by, 
\begin{align*}
&\Delta_{i}\bigg\lceil\dfrac{\log{(\psi T\epsilon_{m_{i}}^{2})}}{2\epsilon_{m_{i}}}\bigg\rceil
\leq\Delta_{i}\bigg\lceil\dfrac{\log{(\psi T(\dfrac{\Delta_{i}}{4\sqrt{2}})^{4})}}{(\dfrac{\Delta_{i}}{4\sqrt{2}})^{2}}\bigg\rceil \text{, since } \sqrt{2\epsilon_{m_{i}}} < \dfrac{\Delta_{i}}{4}\\
&\overset{(a)}{\leq}\Delta_{i}\bigg(1+\dfrac{32\log{(\frac{T}{K^2} T(\Delta_{i})^{4})}}{\Delta_{i}^{2}}\bigg)
\leq\Delta_{i}\bigg(1+\dfrac{32\log{( \frac{T\Delta_i^2}{K})}}{\Delta_{i}^{2}}\bigg)
\end{align*} 
 
In the above case, $(a)$ is obtained by substituting the values of $\psi$ and $\rho_a$. Summing over all arms in $A^{'}$ the total regret is given by, 
\begin{align*}
\sum_{i\in A^{'}}\Delta_{i}\bigg(1+\dfrac{32\log{( \frac{T\Delta_i^2}{K})}}{\Delta_{i}^{2}}\bigg)
\end{align*}

\subsubsection*{Case $b2$: \textit{Optimal arm ${*}$ is eliminated by a sub-optimal arm  }}


	Firstly, if conditions of Case $a$ holds then the optimal arm ${*}$ will not be eliminated in round $m=m_{*}$ or it will lead to the contradiction that $r_{i}>r^{*}$. In any round $m_{*}$, if the optimal arm ${*}$ gets eliminated then for any round from $1$ to $m_{j}$ all arms ${j}$ such that $m_{j}< m_{*}$ were eliminated according to assumption in Case $a$. Let the arms surviving till $m_{*}$ round be denoted by $A^{'}$. This leaves any arm $a_{b}$ such that $m_{b}\geq m_{*}$ to still survive and eliminate arm ${*}$ in round $m_{*}$. Let such arms that survive ${*}$ belong to $A^{''}$. Also maximal regret per step after eliminating ${*}$ is the maximal $\Delta_{j}$ among the remaining arms ${j}$ with $m_{j}\geq m_{*}$.  Let $m_{b}=\min\lbrace m|\sqrt{2\epsilon_{m}}<\dfrac{\Delta_{b}}{4}\rbrace$. Hence, the maximal regret after eliminating the arm ${*}$ is upper bounded by, 
\begin{align*}
&\sum_{m_{*}=0}^{max_{j\in A^{'}}m_{j}}\sum_{i\in A^{''}:m_{i}>m_{*}}\bigg(\dfrac{2}{(\psi  T\epsilon_{m_{*}})^{\rho_{a}}} \bigg).T\max_{j\in A^{''}:m_{j}\geq m_{*}}{\Delta}_{j}\\
%%%%%%%%%%%%%%%%%%%%%%%%%%%
&\leq\sum_{m_{*}=0}^{max_{j\in A^{'}}m_{j}}\sum_{i\in A^{''}:m_{i}>m_{*}}\bigg(\dfrac{2}{(\psi  T\epsilon_{m_{*}})^{\rho_{a}}} \bigg).T.4\sqrt{2}\sqrt{\epsilon_{m_{*}}}\\
%%%%%%%%%%%%%%%%%%%%%%%%%%
&\leq\sum_{m_{*}=0}^{max_{j\in A^{'}}m_{j}}\sum_{i\in A^{''}:m_{i}>m_{*}}8\sqrt{2}\bigg(\dfrac{T^{1-\rho_{a}}}{\psi^{\rho_{a}}\epsilon_{m_{*}}^{\rho_{a}-\frac{1}{2}}} \bigg)\\
%%%%%%%%%%%%%%%%%%%%%%%%%%
&\leq\sum_{i\in A^{''}:m_{i}>m_{*}}\sum_{m_{*}=0}^{\min{\lbrace m_{i},m_{b}\rbrace}}\bigg(\dfrac{8\sqrt{2}T^{1-\rho_{a}}}{\psi^{\rho_{a}}2^{-(\rho_{a}-\frac{1}{2})m_{*}}} \bigg)\\
%%%%%%%%%%%%%%%%%%%%%%%%%%
&\leq\sum_{i\in A^{'}}\bigg(\dfrac{8\sqrt{2}T^{1-\rho_{a}}}{\psi^{\rho_{a}}2^{-(\rho_{a}-\frac{1}{2})m_{*}}} \bigg)+\sum_{i\in A^{''}\setminus A^{'}}\bigg(\dfrac{8\sqrt{2}T^{1-\rho_{a}}}{\psi^{\rho_{a}}2^{-(\rho_{a}-\frac{1}{2})m_{b}}} \bigg)\\
%%%%%%%%%%%%%%%%%%%%%%%%%%
&\leq\sum_{i\in A^{'}}\bigg(\dfrac{4T^{1-\rho_{a}}*2^{\rho_{a}-\frac{1}{2}}}{\psi^{\rho_{a}}\Delta_{i}^{8\sqrt{2}\rho_{a}-1}} \bigg)+\sum_{i\in A^{''}\setminus A^{'}}\bigg(\dfrac{8\sqrt{2}T^{1-\rho_{a}}*2^{\rho_{a}-\frac{1}{2}}}{\psi^{\rho_{a}}b^{2\rho_{a}-1}} \bigg)\\
%%%%%%%%%%%%%%%%%%%%%%%%%%
&\leq\sum_{i\in A^{'}}\bigg(\dfrac{T^{1-\rho_{a}}2^{\rho_{a}+\frac{7}{2}}}{\psi^{\rho_{a}}\Delta_{i}^{2\rho_{a}-1}} \bigg)+\sum_{i\in A^{''}\setminus A^{'}}\bigg(\dfrac{T^{1-\rho_{a}}2^{\rho_{a}+\frac{7}{2}}}{\psi^{\rho_{a}}b^{2\rho_{a}-1}} \bigg)\\
%%%%%%%%%%%%%%%%%%%%%%%%%%
&\overset{(a)}{\leq}\sum_{i\in A^{'}}16K +\sum_{i\in A^{''}\setminus A^{'}} 16K\\
%& = \sum_{i\in A^{'}}\bigg(\dfrac{ C_{2}(\rho_{a}) T^{1-\rho_{a}}}{\Delta_{i}^{4\rho_{a}-1}} \bigg)+\sum_{i\in A^{''}\setminus A^{'}}\bigg(\dfrac{C_{2(\rho_{a})}T^{1-\rho_{a}}}{b^{4\rho_{a}-1}} \bigg) \text{, where } C_2(x) = \frac{2^{2x+\frac{3}{2}}}{\psi^{x}}
\end{align*}

Again $(a)$ is obtained by substituting the values of $\psi$ and $\rho_a$. Summing up \textbf{Case a} and \textbf{Case b}, the total regret till round $m$ is given by,
\begin{align*}
 \E[R_{T}] \leq &\sum\limits_{i\in A:\Delta_{i} > b} \left\lbrace 12K + \bigg(\Delta_{i}+\dfrac{32\log{(\frac{T\Delta_i^2}{K})}}{\Delta_{i}}\bigg) + 16K\right\rbrace +\sum\limits_{i\in A:0 < \Delta_{i}\leq b} 16K + \max_{i\in A:\Delta_{i}\leq b}\Delta_{i}T
\end{align*}
\end{proof}


\section{Proof of Corollary 1}
\label{App:Proof:Corollary:1}
\begin{proof}
%As stated in \cite{auer2010ucb}, the regret bound can be of the order of $\sqrt{KT\log K}$ in non-stochastic MAB setting. This is shown in Exp4\cite{auer2002nonstochastic} algorithm. 
First we recall the definition of Theorem \ref{Result:Theorem:1} below,
\begin{align*}
&\E [R_{T}]\leq 
\sum\limits_{\substack{i\in A_{s^{*}},\\\Delta_{i} > b}}\bigg\lbrace \Delta_{i} + 12K
+ \frac{32\log{(\frac{T\Delta_i^2}{K})}}{\Delta_{i}} \bigg\rbrace
 + \! \! \sum\limits_{\substack{i\in A,\\\Delta_{i} > b}} \bigg\lbrace 2\Delta_{i} +
12K + \frac{64\log{(\frac{T\Delta_i^2}{K})}}{\Delta_{i}} \bigg\rbrace \\
%%%%%%%%%%%%%%%%%
&+ \sum\limits_{\substack{i\in A_{s^{*}},\\ \Delta_{i} > b}} 
16K+\sum\limits_{\substack{i\in A_{s^{*}},\\0 < \Delta_{i}\leq b}} 16K + \sum_{\substack{i\in A\setminus A_{s^*}:\\\Delta_{i}> b}}32K +\sum_{\substack{i\in A \setminus A_{s^*}:\\ 0 < \Delta_{i} \leq b}}32K 
 \!+\! \max\limits_{i:\Delta_{i}\leq b}\Delta_{i}T
\end{align*}

Now we know from \cite{bubeck2011pure} that the function $x\in [0,1]\mapsto x\exp(-Cx^2)$ is  decreasing on $\left[\dfrac{1}{\sqrt{2C}},1\right ]$ for any $C>0$. So, taking $C=\left\lfloor \dfrac{T}{e}\right\rfloor$ and by choosing  $\Delta_{i}=\Delta=\sqrt{\dfrac{K\log K}{T}}>\sqrt{\dfrac{e}{T}}$ for all ${i:i\neq *}\in A$ and substituting $p=\left\lceil \dfrac{K}{\log K}\right\rceil $ in the bound of ClusUCB we get,

	\begin{align*}
	\sum_{i\in A_{s^{*}}:\Delta_{i} > b} 12K =12\dfrac{K^2}{p}
	\end{align*}		
	 Similarly, for the term, 
	 \begin{align*}
	 \sum_{i\in A:\Delta_{i} > b} 12K = 12 K^2
	 \end{align*}
	 
	
	For the term regarding number of pulls,
	\begin{align*}
	\sum_{i\in A:\Delta_{i} > b}\dfrac{64\log{(\frac{T\Delta_i^2}{K})}}{\Delta_{i}} &\leq  \dfrac{64K\sqrt{T}\log{(T\dfrac{K\log K}{T K})}}{\sqrt{K\log K}} \leq  \dfrac{64\sqrt{KT}\log{(\log K)}}{\sqrt{\log K}}\\
	%%%%%%%%%%%%%%%%%%%%%%%
	&\overset{(a)}{\leq} 64\sqrt{KT}
	\end{align*}		
	
	Here $(a)$ is obtained by the identity $\dfrac{\log\log K}{\sqrt{\log K}} < 1$ for $K\geq 2$. Lastly we can bound the error terms as, 
	\begin{align*}
	\sum\limits_{i\in A_{s^{*}}:0\leq\Delta_{i}\leq b} 16K =\dfrac{16K^2}{p} \overset{<}{(a)} 16K\log K
	\end{align*}	 	
 	Here we obtain $(a)$ by substituting the value of $p$. Similarly for the term,
 	\begin{align*}
 	\sum_{i\in A\setminus A_{s^*}: \Delta_{i} > b} 16K =\dfrac{16K^2}{p} < 16K\log K
	\end{align*} 	
	Also, for all $b\geq \sqrt{\dfrac{e}{T}}$,
	\begin{align*}
 	\sum_{i\in A\setminus A_{s^*}: 0 < \Delta_{i} \leq b} 32K = \left(K-\dfrac{K}{p}\right) 32K
	\end{align*} 	
	
	Now, $K-\dfrac{K}{p}= K\left( \dfrac{p-1}{p} \right) < K\left(  \dfrac{\frac{K}{\log K}+1-1}{\frac{K}{\log K}+1 }\right) < \dfrac{K^2}{K+\log K}$. So, after substituting the value of $p=\left\lceil \dfrac{K}{\log K} \right\rceil$, we get,
	
	\begin{align*}
 	\sum_{i\in A\setminus A_{s^*}: 0 < \Delta_{i} \leq b} 32K = \left(K-\dfrac{K}{p}\right)32K < \dfrac{32 K^3}{K+\log K}
	\end{align*} 	
	
	Summing up all the contribution from the individual cases as shown above, the total gap-independent regret is given by,	
	
	\begin{align*}
	\E[R_{T}]\leq & 12K\log K + 32\sqrt{KT} + 12K^2 + 64\sqrt{KT} + 32K\log K  \dfrac{64 K^3}{K+\log K}
	\end{align*}
 	
	So, the total bound for using both arm and cluster elimination cannot be worse than,
	
	\begin{align*}
	\E[R_{T}]\leq 96\sqrt{KT} + 12K^2 + 44K\log K + \dfrac{64 K^3}{K+\log K}\\ 
	\end{align*}		
\end{proof}

%\section{Why Clustering?}
%\label{App:E}
%
%In this section we want to specify the apparent use of clustering. The error bounds are shown in Table \ref{App:E:table:3}.
%
%\begin{table}[!h]
%\caption{Error Bound}
%\label{App:E:table:3}
%\begin{center}
%\begin{tabular}{p{1.4cm}p{10.3cm}p{3.5cm}}
%\multicolumn{1}{c}{\bf Elim Type} &\multicolumn{1}{c}{\bf Error Bound} &\multicolumn{1}{c}{\bf Remarks} \\
%\hline \\
%Only Arm Elimination (ClusUCB-AE)	& \begin{align*}\underbrace{\sum_{i\in A:\Delta_{i} > b}\bigg(\dfrac{C_{2}(\rho_{a})T^{1-\rho_{a}}}{\Delta_{i}^{4\rho_{a} -1}} \bigg)}_{\text{Case b2, Proposition \ref{proofTheorem:Prop:1}}} + \underbrace{\sum_{i\in A:0 < \Delta_{i}\leq b}\bigg( \dfrac{C_{2}(\rho_{a})T^{1-\rho_{a}}}{b^{4\rho_{a} -1}} \bigg)}_{\text{Case b2, Proposition \ref{proofTheorem:Prop:1}}}\end{align*}  & With $\rho_{a}=\frac{1}{2},$ and $\psi=\frac{T}{196 \log K}$ this gives $300\sqrt{KT}+300\sqrt{KT\log K}$. Hence, this has an order of $O(\sqrt{KT\log K})$.\\
%\hline\\
%%%%%%%%%%%%%%%%%%%%%%%%%%%%%%%%%%%%%%%%%%%%%%%%%%%%%%%%%%%%%%%%%%%%%%%%%%%
%%%%%%%%%%%%%%%%%%%%%%%%%%%%%%%%%%%%%%%%%%%%%%%%%%%%%%%%%%%%%%%%%%%%%%%%%%%
%Arm \& Cluster Elimination (ClusUCB) 	& \begin{align*}  \underbrace{\sum_{i\in A_{s^{*}}:\Delta_{i} > b}\bigg(\dfrac{C_{2}(\rho_{a})T^{1-\rho_{a}}}{\Delta_{i}^{4\rho_{a}-1}} \bigg)+ \sum_{i\in A_{s^{*}}:0\leq\Delta_{i}\leq b}\bigg(\dfrac{C_{2}(\rho_{a})T^{1-\rho_{a}}}{b^{4\rho_{a} -1}} \bigg)}_{\text{Case b2, Arm Elim, Theorem \ref{Result:Theorem:1}}}\\   
% + \underbrace{\sum_{i\in A\setminus A_{s^*}:\Delta_{i} > b}\bigg(\dfrac{2C_{2}(\rho_{s})T^{1-\rho_{s}}}{\Delta_{i}^{4\rho_{s}-1}} \bigg)+ \sum_{i\in A\setminus A_{s^*}:0\leq\Delta_{i}\leq b}\bigg(\dfrac{2C_{2}(\rho_{s})T^{1-\rho_{s}}}{b^{4\rho_{s} -1}} \bigg)}_{\text{Case b3+b4, Clus Elim, Theorem \ref{Result:Theorem:1}}} \end{align*} & With $\rho_{a}=\frac{1}{2}$, $\rho_{s}=\frac{1}{2}, p=\lceil \frac{K}{\log K}\rceil$ and $\psi=\frac{T}{196 \log K}$ this gives $\frac{300 \sqrt{T}\log K^{\frac{3}{2}} }{\sqrt{K}} + \frac{300 \sqrt{T}\log K}{\sqrt{K}} + 600 \frac{K}{K+\log K}\sqrt{KT\log K} + 600 \frac{K}{K+\log K}\sqrt{KT}$. So we can reduce the error bound to $O(\frac{K}{K+\log K}\sqrt{KT\log K})$.\\
%\hline
%\end{tabular}
%\end{center}	
%\end{table}
%
%While looking at the error terms in Table~\ref{App:E:table:3}, we see that using just arm elimination (ClusUCB-AE) the elimination error bound is more than using both arm and cluster  elimination simultaneously (ClusUCB). 




\end{document}
